\chapter{User study results}
\label{InformationAchitectureResults}
As it is elaborated on in \autoref{ChapterWorkshop}, the purpose of the card sort and workshop is to elicit mental models of the users' interaction with a TonePrint community platform. In the workshop this is investigated for two specific tasks, which on their own don't provide a full explanation for how the entire information architecture should be. The given tasks are the overall new features going from the current TonePrint app to a TonePrint community, and the results are as such also overall explanations. For further elaborating explanations, the various elements in each task should be investigated on their own as a set of subtasks. The subjects' mental models are derived from their hand drawn concept maps and subsequent explanations of them, which was video recorded, and the purpose of the following chapter is to present the results of this workshop before uniting them with the results of the card sort.

\section{Workshop results}
\label{WorkshopResults}
The workshop was conducted in a meeting room at TC Electronic's headquarters in Aarhus, and five members of their staff served as subjects for this. It's important to note that none of them have been affiliated with either the development or maintenance of the TonePrint application at any point, so they were considered fitting subjects for the workshop. The obvious bias of including developers of the TonePrint app would have meant that the workshop wouldn't have produced mental models representing those of the end-users. One of the subjects, however, was considered a potential bias as he develops templates for User TonePrints. He has not had any influence on the design of the TonePrint app though, so he was included in the individual task any way but instead served as an observer for the group task, leaving the remaining four subjects for this.

For the individual task, the results produced five different hand drawn concept maps, these are all found in \autoref{WorkshopConceptMaps}. The results for the subjects are presented individually including their concept map, depicted with software to make it more presentable and readable, a table of their highlighted content and actions, and codings of their video presentations. For the depicting of the concept maps it was considered insufficient to create a complete remake of their drawings alone, as they differ greatly in terms of structure and detail. In order to balance these differences, the new concept maps are made from their presentation of their drawings. The codings of their explanations are presented with the concept map with the important content and actions highlighted with \textbf{bold text}. This is done to put the content and actions into context and not only have them displayed in a table.

\subsection*{Subject 1}
\label{Subject1ICMM}

\begin{figure}[H]
	\centering
	\includegraphics[width=\textwidth]{1ICMM_new.pdf}
	\caption{The Concept map created by subject 1.}
	\label{fig:ICMM1}
\end{figure}

He begins at the \textbf{TP App} from which he wants to \textbf{Log in} If he isn’t already logged in. Than he enters his site \textbf{My TonePrints} which contains the buttons \textbf{Friends} and /or \textbf{Community}. He uses friends to either enter a \textbf{Search Friends} where he can find specific friends, or he enters \textbf{Friends List} where he gets an overview of all of his friends. He uses this to find/select user 1, hence entering \textbf{User 1 Profile}. On this profile he is presented a \textbf{List of TonePrints} created by user 1. The list is divided in \textbf{By Name} and \textbf{By Type}. From By Type he selects the Corona Type and is presented with a \textbf{Corona TP List}. From this list he’s able to use a checkbox saying \textbf{My List} which adds the TonePrint to his My TonePrints. There’s also a Gear/settings buttons, which he doesn’t know what does. He choses the \textbf{Warm Corona} TonePrint by clicking on the name, which presents him the \textbf{User 1 TP: Warm Corona} site. From this site he has the option to \textbf{Like} the TP, to se \textbf{Number of users using this} and read a description of \textbf{How I use It}.


\subsubsection{Subject 2}
\label{Subject2ICMM}
\begin{figure}[H]
	\centering
	\includegraphics[width=0.4\textwidth]{2ICMM_new.pdf}
	\caption{The Concept map created by subject 2.}
	\label{fig:ICMM2}
\end{figure}

He begins by opening the \textbf{App}. He than have to opportunities, either he would open for his \textbf{Subscriptions} which is people he already subscribes, where he can find user 1, or he could go to another \textbf{list and search} for user 1. This list can be used for finding users who you don’t subscribe to. He then would Connect his pedal and guitar so that he can test the TP he’s about to find. He than go and chooses one of \textbf{User 1's Corona TonePrints} and \textbf{Loads it to his pedal}. He then tries the TP By playing he guitar. If he doesn’t like it, he would go find another. If he likes it he would like to have the option to \textbf{Subscribe to user 1} if he doesn’t already are a subscriber. He would also like the option to \textbf{Add the TonePrint as favorite}, so that it would be easier to find later. He than \textbf{Store the TonePrint} in his pedal. When he chooses to subscribe to user 1 he would like to have the option to select \textbf{Getting notification} either by mail or in app, when user 1 makes something new.

\subsubsection{Subject 3}
\label{Subject3ICMM}
\begin{figure}[H]
	\centering
	\includegraphics[width=0.4\textwidth]{3ICMM_new.pdf}
	\caption{The Concept map created by subject 3.}
	\label{fig:ICMM3}
\end{figure}

The first thing he does is to open the \textbf{TP App}. When it opens I gets a \textbf{Notification} telling that user 1, who he \textbf{subscribes} to have uploaded a new Chorus TonePrint. He than taps on the notification, which sends him to the \textbf{Specific Chorus TP Site}, in which he selects to \textbf{Save TonePrint}. He then gets the opportunity go to \textbf{User 1 Profile} where he can check the rest of \textbf{User 1’s TonePrints}, which he expect is given a \textbf{Logical indexation} which could be alphabetical or something else.

\subsubsection{Subject 4}
\label{Subject4ICMM}
\begin{figure}[H]
	\centering
	\includegraphics[width=0.4\textwidth]{4ICMM_new.pdf}
	\caption{The Concept map created by subject 4.}
	\label{fig:ICMM4}
\end{figure}

He begins by opening the \textbf{TP App} where expect it to have something like the currently infra structure. He navigates to the \textbf{User TonePrints} where he expects to find all of his own User TonePrints and the option to choose \textbf{Cloud Library} in which all the User Toneprints is stored. He chooses the Cloud Library because he wants to find a TonePrint from another user. Then goes to a \textbf{Filter system} where he wants to \textbf{Filter by product} where he chooses to \textbf{filter by Corona}. This presents him a list containing all \textbf{Cloud based Corona TonePrints}. Then he would be able to \textbf{Search for User names} in a \textbf{list} until he finds User 1. He also mentions another approach where instead of filtering use a \textbf{Search Function} to find all of \textbf{User 1’s TonePrints}. After his filtering and searching on User 1 he enters \textbf{User 1’s profile site} Containing all of his TonePrints, however just the ones for Corona, given the filtering. In this \textbf{User 1 Corona TonePrint list} he clicks on one so that he can try it, indicating he is \textbf{Transferring the TP to his pedal}. After trying it he then have the option to save it to his own \textbf{User Toneprint Library} and he’s in doubt what happens if he leaves the TonePrint, so he would also like to \textbf{Add it to Favorites}. He wants a interface like the current where he can \textbf{Save the TonePrint} and \textbf{Store it in pedal}.

\subsubsection{Subject 5}
\label{Subject5ICMM}
\begin{figure}[H]
	\centering
	\includegraphics[width=0.4\textwidth]{5ICMM_new.pdf}
	\caption{The Concept map created by subject 5.}
	\label{fig:ICMM5}
\end{figure}

He wants the app to be simple and similar to some popular apps, so that it’s like something you have tried before. In the \textbf{TP App} he imagines a magnifying glass in the corner so that he can \textbf{Search}. He want to create a option for \textbf{Advanced Search}, where you can personalize what you are searching for, but also that the standard Search is a \textbf{Meta Search option}. This means that it search on ever thing int the system(TonePrints, User, Product…). He imagines that \textbf{Icons could indicate the category} of the search items. So when he searches User 1 it would \textbf{suggest User: User 1}. Then he comes to the \textbf{Public TonePrints by User 1}, which is \textbf{Lists organized by products}. From here he chooses a \textbf{Corona TP} from the \textbf{Corona list}. When selecting the TonePrint he’s presented with the \textbf{Toneprints details}, a \textbf{Picture} that has been attach and a \textbf{description}. If he just Right click or on the phone Hold he’s shown a \textbf{Context menu} where he can \textbf{Beam} the TonePrnt to his pedal, \textbf{Clone} it to his \textbf{Personal Library} where he can \textbf{edit it}, or \textbf{Favorite it} so it goes to his list of favorites. He want’s a \textbf{history function}, so even though he doesn’t save or favorites, he still have the option to backtrack to it, like in a web browser. 

\subsection{Shared Analysis}
\label{SharedAnalysis}
%
The next phase is the shared analysis in which the codes derived from the ICMM's is used, in order to find items (concept and links) which is shared between them. In order to determine which items that identifies as shared a criterion needs to be defined. As described in \autoref{ACSMM} it's suggested to set the criterion at 50\% at the beginning, which afterwards may be adjusted. This means that a given item have to occur in atleast in 50\% of the ICMM's to be identified as shared.\\
To compare the codes they are first typed into tables, for the purpose of gathering the codes for easier overview. The tables containing the codes is \autoref{tab:Subject1Coded} to \autoref{tab:Subject5Coded}. 

\begin{table}[]
	\centering
	\begin{tabular}[width=\textwidth]{c|lllllc}
\hline 
Shared item & subject 1 & subject 2 & subject 3 & subject 4 & subject 5 & \%	\\ \hline
TP App & TP App & App & TP App & TP App & TP App & 100\%	\\ 
My User Toneptints & My TonePrint site &  &  & User TonePrints \& Own User TonePrint Library & Personal Library & 60\% \\
Search function & Search Friends & Search in List & & Search User Name \& Search Function & Search, Advanced search \& Meta Search Option & 80\% \\
User 1 Profile & User 1 Profile & & User 1 Profile & User 1 Profile site & & 60\% \\
User 1 TonePrints & User 1 TonePrint list & & User 1's Toneprints & User 1 TonePrints
	\end{tabular}
\end{table}

The lists containing the subjects individual codes, \autoref{tab:Subject1Coded} to \autoref{tab:Subject5Coded} are compared to each other to find the concepts which is shared. As recommended by \textcite{WEB:ConceptMapAnalysis} is the criterion for a concept to be identified as shared defined as 50\%. The concepts which is defined as shared is listed at \autoref{tab:SharedConcept}(WHich isn't finish). 

It's deemed problematic to use the described steps from \textcite{WEB:ConceptMapAnalysis} to create a ACSMM. This is because the task of the workshop was for the subjects to solve a task with a they have to build up as they go. This gives the users full control in regards to which features that exists in the system, how they works and when they wants to use them. This results in the subjects going different ways around the task and solving it in their won way. This is quite similar to some system in the real world which give you the option to reach a certain goal different ways. This however creates one big problem regarding, them only describing the part of the system, which they uses. This makes it very difficult to compare one persons ICMM with another, because they may have chosen different ways of solving the task and theirby having a minimal of similar concepts in common. This doesn't necessarily mean that the two subject disagree on how the system should work and their mental models might be similar towards the system. They just have chosen different approaches for the task, ehich means that they don't uses the same branches of the system. \\


\begin{table}[H]
	\centering
	\begin{tabular}[width=\textwidth]{|l|c|}
	\hline
	Shared concept & Sharedness level (\%) \\ \hline
	TP App & 100\% \\ 
	User TonePrint Library & 60\% \\ \hline
	\end{tabular}
	\caption{Shared concepts and sharedness level}
	\label{tab:SharedConcept}
\end{table}



\begin{table}[H]
\begin{minipage}[b]{0.5\linewidth}\centering

    \begin{tabular}{|l|}
    \hline
    Subject 1 \\ \hline
    TP App  \\
    Log in  \\
    My TonePrint site  \\
    Friends button  \\
    Community button  \\
    Search Friends  \\
    Friends List  \\
    User 1 Profile  \\
    User 1 TonePrint List  \\
    List by Name  \\
    List by Type  \\
    Corona TonePrint List  \\
    My List Check-box  \\
    Warm Corona TonePrint  \\
    User 1 TP: Warm corona site  \\
    Like  \\
    Number of users using this  \\
    How i use it  \\ \hline
    \end{tabular}
    \caption{List of subject 1's coded concepts}
    \label{tab:Subject1Coded}
    
	\begin{tabular}{|l|}
    \hline
    Subject 2 \\ \hline
    App \\
    Subscriptions  \\
    Search in List  \\
    User 1's Corona TonePrints  \\
    Load to Pedal  \\
    Subscribe to user 1  \\
    Add TonePrint to favorites  \\
    Store in Pedal  \\
    Getting notification  \\ \hline
    \end{tabular}
	\caption{List of subject 2's coded concepts}
    \label{tab:Subject2Coded}    
    
    \begin{tabular}{|l|}
    \hline
    Subject 3 \\ \hline
    TP App \\
    Notification  \\
    Subscribes  \\
    Specific Chorus TP site \\
    Save TonePrint \\
    User 1 Profile \\
    User 1's TonePrints \\
    Logical Indexation  \\ \hline
    \end{tabular}
	\caption{List of subject 3's coded concepts}
    \label{tab:Subject3Coded}
    
\end{minipage}
\begin{minipage}[b]{0.49\linewidth}\centering

    \begin{tabular}{|l|}
    \hline
    Subject 4 \\ \hline
    TP App \\
    User TonePrints \\
    Cloud Library \\
    Filter system \\
    Filter by product \\
    Filter by Corona \\
    Cloud based Corona TonePrints \\
    Search User Name \\
    Users List \\
    Search function \\
    User 1's TonePrints \\
    User 1 profile site \\
    User 1 Corona TonePrint \\
    Transfer to pedal \\
    Own User TonePrint Library \\
    Add to Favorites \\
    Save TonePrint \\
    Store in pedal  \\ \hline
    \end{tabular}
	\caption{List of subject 4's coded concepts}
    \label{tab:Subject4Coded}    
    
	\begin{tabular}{|l|}
    \hline
    Subject 5 \\ \hline
    TP App \\
   	Search  \\
    Advanced Search  \\
    Meta Search option  \\
    Icons indicating category  \\
    Suggest User: user 1  \\
    Public TonePrints by User 1  \\
    List organized by Products  \\
    Corona TonePrint \\
    Corona List \\
    TonePrints Details \\
    Picture \\
    Description \\
    Context Menu \\
    Beam \\
    Clone \\
    Personal Library \\
    Edit \\
    Favorite it \\
    History function  \\ \hline
    \end{tabular}
	\caption{List of subject 5's coded concepts}
    \label{tab:Subject5Coded}
        
\end{minipage}
\end{table}













