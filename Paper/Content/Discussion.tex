\chapter{Discussion}
\label{Discussion}
The following chapter intends to discuss the results of the user study in correlation with comparing the results of the workshop and the card sort. Through this discussion it will be reflected upon how the findings might be used in the design process of the TonePrint sharing platform. Furthermore it would be discussed how this study leads to further studies. In the end the intention is to discuss the entire project in by comparing the research questions in (Research questions).


\section{Takeaways from user study}
\label{UserStudyTakeaway}
%
As described in \autoref{ChapterWorkshop} is the scope of this user study to explore the users mental model of the TonePrint sharing platform. Which is an important aspect of designing the information architecture. In this pursue, two methods is used separately in order to explore how users group the concepts and functionalities of the platform and how they work together in order use the platform. In a perfect world could the results from this study indicate which content and concepts that should be present in different sites, in order to accommodate the users journey to complete a task. However isn't the world always perfect and the results might be to divers in order to be able to decide on a finished IA yet.\\
\\
\subsection{Comparing card sort and individual results}
\label{ComparingIndividualAndCard}
When looking at \autoref{UseOfCardSortData}, \autoref{IndividualTaskReflection} and \autoref{GroupTaskResults} is it very clear that the success of the two methods are very different. The results of the card is badly affected by the small sample size and lac of qualitative data, whereas the workshop provided a fair amount of data both from the individuals models which to some extend could be compared, and the discussion when agreeing on a model.\\
The scope of the individual task in the workshop was that the subjects had to find a TonePrint created by a fictional user. When comparing this task with the results of the card sort the starting point would be to look at the look at the groups focusing on "Finding TonePrints" and the cards representing the concepts and components, that the subjects uses in the workshop. Ass seen in
\autoref{IndividualTaskReflection} and the individual models \autoref{fig:ICMM1} to \autoref{fig:ICMM5} is the first step to either use a search function or filtering in a library or, in order to locate the corona toneprints created by 'User 1'. When looking at cards representing this such as \textit{TonePrint library}, \textit{Searching system} and \textit{'Effect type' filter} it's seen that they are located in the searching groups. For the aspect of filtering and searching on another user is the cards \textit{User library} and \textit{'User' filter} the best representatives whereas only the filter is seen as a searching tool. Bother of these is however grouped as a 'user feed' or 'User Profile', which is concepts that in some way also appears in the workshop results. There has not been provided cards that describes the use other users toneprints or in other ways is directly comparable with the aspects to which the 'User 1 profile' is used to find a 'User 1 corona TonePrint'. However does the two 'Profile', the 'User setting' and 'User feed' groups indicate that there should be some sort of user profile incorporated in the sharing platform. After locating the target TonePrint identifying it as a favorite is often the next step followed be downloading and beaming the TonePrint. While the \textit{TonePrint download system} and \textit{Beaming} cards are sorted in together identifying as a TonePrint software tools, the Favorite part is more difficult to compare, because no card or group is directly referring to the concept. The cards best most related to this is \textit{"Other users also liked this" recommendations}, \textit{'Rating' filter}, \textit{'Popularity' filter} and \textit{User raring system}. The first three of thees is always grouped together (When excluding subject 6) however is there little consistency in the groups thy are placed in, while the latter cards grouping have nothing in common that of the other. A comparison between the two methods for the part of favoring or rating a TonePrint to far-fetched.\\
\\
\subsection{Comparing card sort and group results}
\label{ComparingGroupAndCard}
