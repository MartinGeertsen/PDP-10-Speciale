\chapter{Interview Themes}
\label{App:Themes}
The following appendix contains the resulting themes from the thematic analysis of the interview with the 4 members of the development team at TC Electronic. Each theme is presented with a brief elaboration of what it covers, and they are presented in chronological order as they came up during the analysis with only a few exceptions. Some themes were formulated during a second iteration by reorganising some of the codes and by renaming some of the theme names in order to better fit the content. The approach and method for the analysis is described in \autoref{ThemanticAnalysis}.

\subsection*{Beaming app}
\label{App:ThemeBeamingApp}
The focus of this theme is the TonePrint application in its early stage, where the editor wasn't included. Instead, the app only consisted of a library with artists TonePrints that could be filtered by either \textit{guitar} or \textit{bass}. The reason for the label "Beaming App" comes from the four subjects describing the app this way, as beaming already was available through the pickup in the instrument in this early stage. The codes within the theme covers descriptions of different elements of the app, as the first question is interested with the decisions made during the initial design process of the app.

\subsection*{No external inspiration}
\label{App:ThemeNoExternalInspiration}
This is a small theme containing comments from the subjects when they're asked whether they have sought inspiration from any internal or external products during the development of the TonePrint app. And as the name indicates, they generally don't believe that they were inspired by any external ones. However, one of the subjects commented that a sister company had a product with some similarities, but he didn't consider it a source of inspiration.

\subsection*{Development tool}
\label{App:ThemeDevelopmentTool}
The TonePrint editor and general concept stems from a development tool used by TC Electronic for creating parameter settings in new pedals. This theme is concerned with that tool, referred to as \textit{virtual front}, and originally it wasn't something TC had offered customers, as it was rather complicated and primarily utilized by the sound engineers at TC. It was used to set which parameters that was available for editing in the knobs on the pedals, and it has, as already mentioned, since been developed into a product on its own in the form of the current TonePrint editor. 

\subsection*{Parameter design}
\label{App:ThemeParameterDesign}
Given the functionality of TC Electronic's products, the different parameters of the audio settings are set by models described as \textit{meta models}. The meta model is what is altered to define the sound of the given product. A challenge for TC has been to make these models easier to read and understand by the users e.g. how gain is controlled with labels, intervals etc. These models are what the editor is altering, and the underlying challenge for TC is to avoid showing the users all parts of the concept and risk being copied.

\subsection*{Internal hierarchy}
\label{App:ThemeRollAllocation}
This theme deals with the internal structure of the development team, more specifically in correlation with how decisions are made. At some point during the design process, decisions has to be made to reach deadlines, and it was explained by the subjects that the \textit{product manager} has the final saying in these scenarios. This has lead to situations where the developer had to integrate features in the app that he wasn't capable of explaining himself due to high complexity. In other situations, however, the developer was more persistent and managed to get the final saying in how the feature should be designed.

\subsection*{Feature prioritisation}
\label{ThemePrioritizingOfFeatures}
During development of new products, TC have to prioritise which features to include first. All the members of the development team has a saying in this process, but users are also involved in the process, if they have specific feature requests. These requests are gathered from the existing online communities such as \textit{TonePrint Junkies} on Facebook and their own Music Tribe site. When it comes to deciding on which features to implement, two factors play an important role, how easily can it be implemented? and how important is the feature? If a feature is considered important but at the same time time-consuming to implement, other requests may be implemented before it. The TonePrint community is in its own right a request made by many users and is a great example of the just mentioned prioritisation process. It's heavily requested and as such important but also difficult to create. As such, it has since been postponed, while other features have been implemented in the existing TonePrint app.

\subsection*{Decisions made from testing}
\label{App:ThemeDecisionsBasedOnTest}
TC Electronic doesn't have extensive experience with user involvement but have already conducted a few user studies. Some design decisions have been made on the basis of the results in these studies where competitors' products was investigated in order to get into other markets. One example involved a type of pedal, they weren't making themselves at the time. They then decided to study competing products and found that the interface and controls of that pedal were too complicated. They then applied this knowledge to their own pedal which then became more succesfull.
 
 The concept of making the editor available for users, instead of just using it as a development tool, also came from a user study. Here, a group a bass players were given access to the virtual front with the task of creating a new sound for a bass amp, which resulted in a new setting for the TC products. The result of this eye-opener for TC became what is now referred to as the TonePrint editor. Finally, a user workshop was conducted by TC focusing on developing ideas for a TonePrint community \parencite{PDF:BrugerWorkshopUserTonePrints}. This workshop lead to multiple ideas for content and features, but the development of this is still at a conceptual stage.
 
\subsection*{Experience from earlier products}
\label{App:ThemeExperienceFromEarlierProducts}
This theme highlights that TC Electronic utilizes the experience they have gained during the development of some of their earlier products and from their user studies. One example of this is a pedal that turned out to be too complicated and too expensive, and another is the user study on competing products mentioned in the previous theme. This experience has also been present in the development of he TonePrint concept, as they kept it in mind that they had to focus on reducing complexity.
 
\subsection*{Focus on the target users}
\label{App:ThemeFocusOnTargetUsers}
There are different kinds of target users for the TonePrint concept in general. Some of them just wants a regular pedal with physical buttons on them for editing the sound, while others have more extensive requests for content and functionalities. Some of them are users of the TonePrint pedals and application because they want to sound like their idol or want to discover new sounds made by professionals, while others wants to make their own TonePrints from the bottom. The requests from these different groups are met in the form of the pedals themselves, the Artist TonePrints, and the TonePrint editor. These groupings of target users are not something that TC have defined through user studies but more from a gut-feeling. Finally, for the TonePrint community, there also is the group of people who wants to share their TonePrint with each other.

\subsection*{Decisions made from assumptions} 
\label{App:ThemeDecisionsMadeFromAssumptions}
This theme further emphasises how TC don't have much experience with user involvement. The decisions of which features to include and how to design them are mostly based on assumptions of what they believe the users want and what works best. An example of this is the design of the interface for the parameter settings in the TonePrint editor. The developers describe how this was shaped accordingly to their own assumptions, with one of them further stating that he would expect it to simply be the typical way of setting it up.

\subsection*{Decisions made from convenience}
\label{App:ThemeDecisionsMadeFromConvenience}
Here, two situations are mentioned of how convenience played a role in decision-making. An example of this is how they want to implement options such as "link to Youtube" and "like your favourite users" in the TonePrint community through technology that they already possess. Another interesting point made in the interview is that they in some instances have had a tendency to just try out the easier design proposals before discussing what they actually want to achieve from it.

\subsection*{Communication in the development of an app}
\label{App:ThemeCommunicationInTheDevelopmentOfAnApp}
From the interview it was found that the current level of communication in the development and maintenance of the TonePrint app is insufficient. It was found that the platform-specific differences between the different versions of the app to some extend is caused by a different programmer for each platform. They have a level of "artistic freedom" and they are furthermore seated in different teams. As such they have each implemented features depending on what is commonly used on their given platform and communication between them is complicated due to their different seatings.

\subsection*{Business model}
\label{App:ThemeBusinessModel}
The TonePrint concept enables TC to always upgrade their product by adding new content in the form of new artists and templates. In this theme it's also highlighted how TC wants to separate themselves from their competitors which is why they have conducted user tests of competing product.

\subsection*{TonePrint descriptions} 
\label{App:ThemeTonePrintDescription}
For every TonePrint in the library, there is a text description telling something about the artist and the specific TonePrint. The motivation for what is written in these descriptions typically comes from members of marketing who meet up with the artists in question. The aim for the descriptions is to involve the artists and make the users want to try out the given TonePrint. By providing them, the users have some information about the settings in the TonePrint in question, and this is considered further valuable in a TonePrint community. For professional artists, it may seem more obvious for the users what the artists wanted to achieve. Furthermore, in the case where many TonePrints for the same pedal are made, the descriptions will help the users sort out which ones they want to try. there could easily come alot of TonePrints, where these descriptions can help the users sort out which ones they want to try. 

\subsection*{TonePrint Concept}
\label{App:ThemeTonePrintConcept}
This theme covers comments made on the TonePrint concept in general. The TonePrint concept consists of three layers: Regular pedal adjustment, beaming of Artist TonePrints and templates, and finally the creation of User TonePrints. Each of these layers aims at different target users and helps TC expand the spectrum of possible end-users. Many of the comments are concerned with how TonePrints allow the users to sound more like their idols and as such is a unique product. Other comments also covers their thoughts of including a community for sharing user TonePrints in a future version of the platform and what should be included in this.

\subsection*{UI design specifications}
\label{App:ThemeParameterUI}
Going from the former TonePrint editor to the current, a lot of the UI was changed. In the early version, many interactions were controlled with sliders, and this was challenging for the UI designers, as there were problems with mapping these sliders to the screen, especially for the phone users. The comments in this theme is concerned with how they made changes to the editor in order to make it more user friendly. A current challenge being discussed is the lack of information at each parameter setting. There is a lack of cues to what the parameter changes, so the main way to figure this out is to try editing them and listen. one solution proposal up for debate is to group the more similar parameters together and provide descriptions of them. This should enable the users to better understand how to interact with the sliders in order to reach a desired effect.

\subsection*{No user involvement} 
\label{App:ThemeNoUserInvolvement}
This is a very brief theme containing comments of situations where TC haven't involved users in the design process. It's mainly in correlation with the first question concerning how they decided which features to include in the TonePrint app.

\subsection*{Decisions made from personal opinions} 
\label{App:ThemeDecisionsMadeFromPersonalOpinions}
Many decisions made in the design process comes from personal opinions. This theme emphasises the need for more user involvement instead of relying on a general gut-feeling within the development teams. Designing the information architecture of the TonePrint was for example done on the basis of their own gut feeling of how it should look and behave.

\subsection*{Inspiration from external products} 
\label{App:ThemeInspirationFromExternalProducts}
In this theme it becomes clear that TC Electronic have been looking at other companies for inspiration for the TonePrint Community. Examples of this are \textit{Soundmondo} from Yamaha and \textit{The Fyre Effect} from AudeoBox, who both offer a community, to some extend, to the users. From their research into these, they have found that having a 'like' feature may risk many effects being difficult to to discover until they get likes themselves, which TC wants to avoid happening in the TonePrint community.

\subsection*{Micsellaneous} 
\label{App:ThemeMiscellaneous}
As the name of this theme much indicates, it contains codes that are too unique to fit in elsewhere. If the codes don't seem to bring any applicable content to the analysis, they may also be placed in this theme.

\subsection*{Native UI}
\label{App:ThemeNativeUI}
Following the initial release of the newest version of the TonePrint app, TC received negative feedback from users who thought the app was simply working too slow. The reason for this was that TC originally used a cross-platform approach for developing the TonePrint app. Things that made sense for developing the app on android would cause issues on the Iphone version and vice versa. TC therefore made changes and fixed these issues by making different design solutions for the different platforms.

\subsection*{User involvement} 
\label{App:ThemeUserInvolvement}
As it has already been elaborated on, user involvement isn't a completely new area for TC, but it is something they want to apply further in their design process. This theme covers the codes concerned with situations where TC actually did involve the users to some extend, and an example of this is how the development of the TonePrint editor was based on a user study involving bass players interacting with the virtual front. Another example is the workshop conducted by \textcite{PDF:BrugerWorkshopUserTonePrints}, where potential end-users' wishes for a TonePrint community was investigated. It seems reoccurring that they want to know what find out what the users want through these studies instead of figuring out what they need.

\subsection*{External developers} 
\label{App:ThemeExternalDevelopers}
TC have on previous occasions involved external designers and developers in the development products. A great example of this is the very first version of the TonePrint app. This was designed by an external designer but implemented by TC's own programmers.

\subsection*{User feedback} 
\label{App:ThemeUserFeedback}
There are typically two ways that TC receives feedback from their users. Either they get feedback through their facebook site \textit{TonePrint Junkies} or through their Music Tribe Community Forum. Much of the feedback they get is about feature requests for the TonePrint app, and this is something that TC are good at keeping track of. The feedback is also about issues found in their products. Every member of the development team is a member of the facebook site, so each of them do have some user contact, but typically the feedback has been through the marketing team first.

\subsection*{Community requests} 
\label{App:ThemeCommunityRequests}
The idea of being able to share User TonePrints with other users has been a request since the beginning of the TonePrint concept. TC obviously acknowledges this request, but as it has already been mentioned, it is an extensive and time-consuming task. They have however been making steps towards beginning an actual development process of this community, as they see it as a matter of time before the users find other ways of sharing User TonePrints with each other.

\subsection*{Sharing detained} 
\label{App:ThemeDetainSharing}
In correlation with the previous theme, TC have had to make precautions for the users not to find ways of sharing User TonePrints through other means than their own systems. TC describes how they have locked the files that represent the TonePrints in order to stay in control of how this is done. This has lead to some users finding creative workarounds to share their TonePrints anyway, with examples of users who have taken screenshots of the parameter settings in their editor and tell the other users to set their parameters the same way.

\subsection*{Community tags} 
\label{App:ThemeCommunityTags}
One feature that seems to be of great interest for the TonePrint Community is what TC refers to as \textit{tags}. The intention is to assist the users of the TonePrint community in finding certain TonePrints by providing them with tags for categorisation. Some suggest that the users should be completely free to come up with these tags themselves, while others suggest that the users should be provided with a limited number of pre-defined tags. They do however agree that the challenge with such a feature is that it depends on the users' understanding of the tags. What may be obvious for some users may at the same time be misleading to other users, as it can be very subjective what they mean.

\subsection*{Community Features (besides tags)}
\label{App:ThemeCommunityFeaturesNotTags}
Beside the tagging feature, TC has also given thoughts to other elements to include in a future TonePrint community. Some of them emphasise a way of subscribing to another user, where they would be kept updated when this user would upload new TonePrints. This would make the users resemble the professional artist in the existing TonePrint app, as they would have the potential of becoming idols themselves. In order to further support this possible motivation for using the TonePrint community, Text descriptions and other self-promoting elements such as links to examples on Youtube or SoundCloud could also be included. For all of the subjects in the interview, it in general seems as the ideas for the TonePrint community relies on some sort of account system for the users, which currently doesn't exist for the TonePrint concept and as such requires a database for managing it. Other important considerations revolves around how the TonePrints are prioritised when displayed to the users. Some of them suggests by name, rating, or whether they have an extensive text description. One of them even suggests a way of logging which TonePrints are used more then others and then bring these to the top of the list. Whatever the solution may include, they all agree that an effective search functionality should be included with this.

\subsection*{Community decisions} 
\label{App:ThemeCommunityDecisions}
This theme is very broad, as it covers decisions already made for the community in general. One of these, which they don't fully agree on yet, is how much they believe the community should be integrated in the current TonePrint app. Some of them wants it fully integrated with the current system maybe as a tab much like with the editor, Others want to keep it separated with just a few sharing and uploading options available in the app. They want the remaining "social media" elements to be a thing of its own. What they do agree on, however, is to include the users in its development, in order to meet their expectations in the end.

\subsection*{Focus on users and user-friendliness} 
\label{App:ThemeFocusOnUsersAndUserFriendliness}
This theme covers the extend to which they already know their users. They want their users to easily be able to use their products, but they have currently just scratched the surface of including them in the design process. The comments concerning this do state that user-friendliness is a high priority focus for them, but there seems to be a lack of understanding of how and why.

\subsection*{TonePrint app prioritisation} 
\label{App:ThemePrioritizingOfTheTonePrintApp}
The maintenance and further development of the TonePrint app has become a priority on the same level as for other TC Electronic products despite that they offer this app for free. This has resulted in more time and resources being allocated to its further development. This is also a factor in why the development of the TonePrint community has been postponed.

\subsection*{SCRUM}
\label{App:ThemeScrum}
This theme is concerned with their framework in the design process, SCRUM. Each sprint is three weeks long and by the end of this timeframe, each member may present results of their work. Before the start of a sprint, they check the backlog to determine what tasks to focus on and how to prioritise them. If a task is considered too time-consuming for a three week sprint, it is deconstructed into smaller tasks which then are divided between the current and future sprints. The SCRUM method is in general considered beneficial by the development team, as they agree it helps them focus on what is currently at hand and then postpone the less important tasks for later sprints. Planning and executing the sprints seems to be a fine balance between staying realistic in correlation with goals, and having more tasks to engage with if they finish their initial ones before the end of the sprint.

\subsection*{No outlined goals for the TonePrint app} 
\label{App:ThemeNoClearGoalsForTheToneprintApp}
TC don't seem to have outlined goals for the TonePrint app, and in their own words this may be because it's a product they offer for free. Every now and then, some of them do check how many people are using the system though.