\chapter{TonePrint app usability issues}
\label{AppendixHeuristics}
The content of this appendix is the usability problems defined through the heuristic evaluation of the TonePrint application \autoref{SectionHeuristicEvaluation}. The problems are listed in accordance with the heuristic it violates. No problems were identified for the "User control and freedom" and "Help users recognise, diagnose, and recover from errors", which is why the two heuristics aren't mentioned below.

\subsubsection{Visibility of system status}
\begin{itemize}
	\item When browsing through the available TonePrints for artists, some of them may have created the same TonePrint settings for multiple pedals. Clicking between these doesn’t provide any clear feedback to which is selected however, as the description of the TonePrint is the same whichever pedal it is set for.\\
	\item There is a lack of indication to which instrument is selected, as this selection happens in settings and not in the list itself. If either \textit{guitar} or \textit{bass} is selected under the instrument filter, and not \textit{all}, the message in the list \textit{"all TonePrints by…”} is misleading, as the user is only going to find TonePrints for one instruments.\\
	\item When pressing \textit{user} on the computer application there are no indications of what to do next. The user is just presented a blank column with nothing in it.\\
	\item When selecting the \textbf{Helix Phaser} with the \textit{guitar} filter active on the computer app, nothing happens. When trying this on the Iphone app, it opens one TonePrint, and when opening it on an android unit, the app crashes.\\
	\item When pressing the video icon on the android and computer app, it isn't clear that the unit will open youtube in a web browser compared to the Iphone app.\\
	\item There is no indication of which TonePrint you already have stored in the pedal connected.
\end{itemize}
%
\subsubsection{Match between systems and the real world}
\begin{itemize}
	\item The sliders for the various parameters are all presented as circular sliders, but interaction with them are done by pressing the center of it and swiping up or down. As such there is a risk of grabbing the entire canvas and not the parameter in question.\\
	\item It appears to still be possible to select bass TonePrints with the \textit{guitar} filter active.
\end{itemize}
%
%\subsubsection{User control and freedom}
%\begin{itemize}
%	\item Nothing here...
%\end{itemize}
%
\subsubsection{Consistency and standards}
\begin{itemize}
	\item Some artists have published the same TonePrint for multiple pedals and when switching between these, the text description is the same. However, in some cases there is a noticeable difference when doing these switch, as some of the descriptions has minor spelling or typeset errors, even though they should be identical.\\
	\item When opening a video description of a TonePrint with its creator on the smartphone app, it is presented in a new window. When opening one in the computer app, it passes you on to the given video on youtube. \\
	\item When browsing TonePrints, there are different buttons in the top right corner of the description page, depending on on the TonePrint.\\
	\item When watching a video description of a TonePrint on the Iphone app and the user at some point wants to return to the list of TonePrints or artists, it demands two different interactions. First, the user must swipe down in order to return to the TonePrint description, before either swiping right or pressing \textit{back} to get back to the list view.\\
	\item When choosing the \textbf{SpectraComp Bass Compressor} with the \textit{guitar} filter on, the user dosen't get the same menu as when choosing other pedals. This is probably due to it being a bass effect.\\
	\item When creating a favourites list, the TonePrints are sorted by pedal name, even if the user selects \textit{sort by artist}.\\
	\item When opening the app on an android unit, the user gets informed that he needs a midi connection. This message doesn't appear on the desktop version, even though the same goes for that.\\
	\item The user has a search functionality available on the android system but not on either the desktop or Iphone version.\\
	\item After connecting the pedal and setting the app to ‘browse by artist’, the artists who have a TonePrint applicable for the pedal aren’t highlighted. This contradicts what happens if the app is set to Brows by Pedal.\\
	\item After you have created your own TonePrint and saved it, it becomes visible in the user menu and three dots to the right of it indicates interaction possibilities. On a mac and as a mac user it’s expected that you can swipe on the computer’s trackpad, instead of using the three dots, this is however not true in this case. 
\end{itemize}
%
\subsubsection{Error prevention}
\begin{itemize}
	\item The typical confirmation dialogue of either \textcolor{xGreen}{\textbf{$\checkmark$}} or \textcolor{xRed}{\textbf{$\times$}} is presented to the users with these icons inside the button on the Iphone app. As such it isn’t clear whether the user selects an action when it is visible, or if this visibility means that it is already selected.\\
	\item When the user is beaming a TonePrint to the pedal, he is given the instruction: \textit{If your pedal flashed like this beaming was a succes}. In order to follow this instruction the user would have to focus on the pedal, and by doing this he wouldn't have seen this instruction in the first place. As such, the user has to focus on two things at once.\\
	\item The user can assign different parameters to the same physical button on the pedal, allowing for live editing of the TonePrint. However, the pedal comes with a print above the knob on the pedal itself, which can't change. As such, the user can potentially edit a parameter, even though the knob says something different.\\
	\item In the mapping settings of the editor part it’s possible to select which parameters the physical buttons of the pedal should affect and map their behavior. When changing the parameters, the mapping of the new parameters is set to a default. If you regret and wants to use the former parameter as it was mapped you would have to remember how it was mapped, because it’s also given the default mapping. \\
	\item Knob range: The indications of what’s represented on the x and y axis in the mapping part of the editor is not very clear. The writing is dark blue on a dark blue background.
\end{itemize}
%
\subsubsection{Recognition rather than recall}
\begin{itemize}
	\item When switching between \textit{browse by product} and \textit{browse by artist}, this has to be done under settings, and the same goes for switching between type of instrument. Instead of having this filtering action visible with the list, the user must remember to check this in the settings menu. 
\end{itemize}
%
\subsubsection{Flexibility and efficiency of use}
\begin{itemize}
	\item In general there are limited ways of customising the canvas, for example the favourite list.\\
	\item The search functionality on the android app only allows for searching in the open menu, making it almost redundant. The user still needs to to go to the right menu before searching for specifics, making scrolling a faster way of finding the right TonePrint. \\
	\item While browsing the TonePrints in the Library menu or the templates menu you are forced to use the “Back arrow”, in the upper left corner, to navigate back to the main page of the menu. This contradicts with the expectations of the menu icons also being applicable for this interaction, which stems from interaction with other systems.
\end{itemize}
%
\subsubsection{Aesthetic and minimalist design}
\begin{itemize}
	\item It’s limited to what extend the size of the canvas can be expanded on the computer app. If it is made full-screen it will no long match the size of the window and take all the space. Instead, the far right of the window will just be a blank column of nothing.\\
	\item When opening the computer application, until something is chosen, the screen will primarily be just blank.
\end{itemize}
%
%\subsubsection{Help users recognise, diagnose, and recover from errors}
%\begin{itemize}
%	\item Nothing here...
%\end{itemize}
%
\subsubsection{Help and documentation}
\begin{itemize}
	\item When choosing \textit{Editor Help}, the user is sent to the main TonePrint webpage.
\end{itemize}
%