\chapter{Interview guide}
\label{AppendixInterviewGuide}
%
This appendix contains the interview guide used for interviewing the members of the development team of the TonePrint app at TC\autoref{Interview}. The interview guide is in Danish because the teams members were all danish. At first there is a small introduction which were read to the interviewee, also ensuring their consent to audio record the interview. This is followed by a list of the predefined questions the interview is based on. 
%
\section{Interview with TC}
\label{InterviewInputs}

\subsubsection{Introduction}
Formålet med dette interview er, at vi gerne vil have et indblik i jeres udviklingsproces af TonePrint appen, da fokusset for vores projekt er at kigge på, hvordan et fremtidigt TonePrint community kan udvikles. Interviewet kommer til at foregå under et semistruktureret format. Det vil sige, at vi har forberedt nogle spørgsmål, men hvis du har nogle pludselige indskydelser eller ekstra informationer, du tænker vil være relevante, så skal du endelig ikke holde dig tilbage med disse.

For at vi kan holde styr på de mange informationer, vi må få ud af interviewet, kunne vi godt tænke os at lydoptage det. I den forbindelse, vil vi selvfølgelig gerne høre, om det er ok med dig? Optagelserne har til formål at hjælpe os videre i processen med vores projekt, og dit navn vil på ingen måde fremgå af vores dokumentation.

\begin{itemize}
  \item Da i udviklede konceptet for TonePrint appen, hvordan besluttede i hvilke funktioner der skulle være med og hvordan de skulle designes?\\
  \item Hvordan har jeres viden angående jeres brugere påvirket udviklingen af TonePrint appen, og hvor har i den viden fra?\\
  \item Gjorde i noget for at målrette TonePrint appen mod bestemte brugergrupper, og hvordan gjorde i det i såfald?\\
  \item Selvom TonePrint appen er et ret unikt produkt har i så draget inspiration fra andre interne eller eksterne produkter, og i så fald hvordan?\\
\end{itemize}

\begin{itemize}
  \item Hvordan besluttede i jer for informationsstrukturen i TonePrint appen, både set i forhold til menustrukturen og de forskellige måder de kan kategoriseres på?\\
  \item I har en meget stor database af både TonePrints, pedaler, kunstnere og videoer. Hvordan besluttede i jer for,  hvordan i håndterer og præsenterer de forskellige data?\\
  \item Hvilken data vil du mene er nødvendig for at kunne gøre et TonePrint community med User TonePrints effektiv, og hvordan vil du mene denne data skal struktureres og kategoriseres.\\
  \item Hvad ligger til grunde for forskellen på appen fra platform til platform? Eksempelvis informationen om ikke tilsluttet pedal, søge funktionen, video visning og TonePrint information samt beaming?\\
  \item Hvad er formålet med tekstbeskrivelserne tilhørende de forskellige TonePrints, og hvordan beslutter i jer for, hvad der skal stå?\\
\end{itemize}

\begin{itemize}
  \item Hvilken type feedback får i vedrørende TonePrint editoren, og hvordan bruger i denne feedback?\\
  \item Til hvilken grad bruger i informationer, i får gennem TonePrint-junkies-facebook-siden, youtube eller music tribe community?\\ 
  \item Meget har ændret sig op til den nuværende app. Hvorfor ændrede i både den grafiske identitet og flere features?\\
  \item Hvilke positive og negative effekter har jeres SCRUM arbejdsmetode haft på udviklingen af TonePrint appen?\\
  \item Hvilke teknologiske begrænsninger har i haft under udviklingen af TonePrint editoren, og hvordan har i kompenseret for disse?\\
  \item Hvordan opstillede i kravene for TonePrint appen, både konceptuelt og design mæssigt?\\
  \item Hvordan opstillede i målsætninger for TonePrint appen? og hvordan sikrede i jer, at disse blev nået?\\
\end{itemize}

\begin{itemize}
  \item Hvis du skulle nævne fem vigtige aspekter som vi bør tage med videre i udviklingen af et TonePrint Community, hvad skulle det så være?
\end{itemize}