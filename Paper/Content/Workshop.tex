\chapter{Workshop}
\label{ChapterWorkshop}
In \autoref{TaskAndMethods} it's decided to conduct a workshop alongside a card sort, focusing on shaping the information architecture and highlighting the users mental models of the TonePrint Community. This chapter describes the design design, planing, execution and evaluation of the workshop.\\

\noindent
The choice of conducting task 1 was decided through meetings with our company supervisor from multiple arguments. These should either be elaborated on in the previous chapter or be the baseline for this. The reasoning behind choosing task 1 comes from not only our own viewpoints, but also those of our supervisor and company supervisor, these are as followed:
\begin{itemize}
	\item It is the next logical step in the development of the TonePrint community, as no real platform currently exists for it. TC are at a very early stage of development, and as such it seems fitting to provide them with an assessment of how they should organize information, content, features, etc. as a baseline for the future design process.
	\item From previous investigations in this project, we have clarified that TC follow the Scrum framework in the shape of 3-week sprints. As such much of the discussion was on fitting the task to a typical sprint. But, since we don't have much time left at this point, we instead choose to fit it to a 2-week sprint. Studying the information architecture through card sorting and workshops is something that can fit to such a time-frame.
	\item Personas representing the typical users of TC products are considered very valuable for TC in their understanding of them. The method is however extensive and much time-consuming to execute and definitely not something that can be done during a 2-week sprint. It was therefore opted out for this project, as it is considered a semester project on its own. If we were to execute it anyway, we would only be able to acquire demographical info on the users, which isn't considered very valuable for TC.
	\item Task 3 and 4 could just as well fit a 2-week sprint, but each of them require an existing app to some extend. It doesn't necessarily have to be a fully functional system, but still a version 
\end{itemize}



