\chapter{User Study of The TP Community}
\label{ChapterWorkshop}
The tasks presented in \autoref{TaskAndMethods} serve as proposals for how to include the users as an active part of the design process at TC Electronic in correlation with developing a platform for sharing User TonePrints. The tasks are all considered equally important, but ultimately only one of them is explored in this project due to the restricted timeframe. Through discussions with our company supervisor, task 1 is chosen for this purpose. The following chapter will start with an elaboration on this decision before moving on to the execution and subsequent analysis of the results, including what methods are applied for both parts.

\section{Deciding on a task}
\label{TaskDecision}
As described in \autoref{Task1}, the purpose of task 1 is to investigate how to design the information architecture of the TonePrint community, in order to ensure that the design accommodates the the users' mental model of the system. Many considerations were put in to the decision of conducting this task, including TC's wishes, how well it would fit into a typical SCRUM sprint, the remaining timeframe for this project etc. The impact of all these factors also influenced what methods to consider and how to approach them.

Firstly, an investigation of the information architecture seemed the most fitting to conduct due to the current state of the TonePrint community. TC are at a very early stage of development, as no version of a sharing platform currently exist. TC therefore considered it evident to start with an assessment of how content and features should be structured as a baseline for the future design process.

Secondly, previous investigations in this project have showed how TC follows the SCRUM framework in the shape of 3-week sprints (see \autoref{ThemanticAnalysis}). As such, much of the discussion was on finding a task and suitable method that could easily fit to this timeframe, and task 1 was here considered ideal. the limited time remaining on the project furthermore required to conduct it as a 2-week sprint, and even though this is not same duration as a sprint at TC, conducting a task in this timeframe would still provide TC insight into how UX tests can fit to a short time period. Task 3 and 4 could just as well fit this timeframe, but each of them required an existing app to some extend. It wouldn't necessarily have to be a fully functional system, but still one that could represent the appearance and functionalities of the system.

Thirdly, creating personas of the typical users of TC products was considered very valuable for TC in their understanding of them. However, this task was too extensive and time-consuming to execute in a 2-week sprint and is in general considered a semester project on its own. If it were to be executed anyway, it would only result in demographical info on the users, which isn't either very valuable for TC.\\

\noindent
When choosing task 1, further specifying the approach is required, as it proposes two different ones. One of these is through Lo-fi testing which is a great tool, but as it is elaborated on in \autoref{Task1}, it requires a rough idea of how the system should work and how the information should be presented. For the TonePrint community there currently is no rough idea of how it should look, and it would as such make more sense to include the users in shaping the concept and flow of the community from the bottom through a \textit{participatory design procedure}. The approach should then follow the outlined suggestion in \autoref{Task1} by conducting a workshop with potential end-users as subjects in order to explore their mental models of a TonePrint community.

\section{Mental Models}
\label{MentalModel}
Before proposing the experimental design and setup of a workshop, it seems appropriate to first define what is meant by \textit{mental models}, and how they are constructed. In \autoref{UXbenefits} it is mentioned that if the developers have the same understanding of how a given product should look and behave, then the users are more likely to have a pleasant experience. this is a quick explanation of the term, as it merely touches the surface of it. In general, mental models are types of conceptual models that resides in people minds, as the name implies. In other words, mental models are the users' understanding of how something works. The definition of a conceptual model is already outlined in \autoref{CommunityConcept} as an explanation of how something works, including all concepts in it, and how they relate to each other. For mental models, different people may hold different versions for the same system, and some people may even hold multiple models of the same system, focusing on different aspects of its operation \parencite[][26]{PDF:DonNorman}. \fxnote{Section is not finished} 
%Mentale modeller er beskrevet nu, så skal jeg bare beskrive, hvordan man kan udvinde den fra brugerne, og hvad man skal være opmærksom på. But that will have to wait a bit...


%\parencite{WEB:NielsenNormanCardSort}\\
%\parencite{WEB:ConceptMapAnalysis}\\

\section{Experimental design}
\label{ExperimentalDesign}
When engaging in task 1, it's important to remember that the main goal is to investigate possible end-users' mental models of a sharing platform for User TonePrints, in order to asses how the information architecture of it should be shaped. The employment of end-users and not generic users is important, as they represent the mental models of the people who are most likely to use a finished version of the system. The current TonePrint app already aims at a certain audience, i.e. guitarists, and employing subjects with no experience with this field wouldn't result in a fitting outcome of the study.

Shaping the information architecture of a system requires multiple steps of generating ideas for content and communicating these, before moving on to testing them with the users \parencite[][356-364]{PDF:InformationArchitecture}. For this task, the first step would as such be to generate ideas for content in a TonePrint community, which features should be included, and how should they communicate with each other? Instead of engaging in a phase of generating such ideas, this paper instead suggests to make use of studies already conducted by TC focusing on a sharing platform for User TonePrints. A study by our company supervisor \textcite{PDF:BrugerWorkshopUserTonePrints}, investigates what is required from such a platform, and the result is given in the shape of a list of content which, if included, would make end-users more interested in using the platform \parencite[][35]{PDF:BrugerWorkshopUserTonePrints}. It's important to note that this content also is produced by the users in a workshop, and by utilizing this information, it will serve as a good starting point. Furthermore, the interview conducted with members of the TC staff previously in this project may also contain appropriate content for developing an appropriate strategy for the information architecture.

As these two sources of inspiration provide meaningful content for a sharing platform for User TonePrints, the next step in eliciting the users' mental models is then to investigate, how the users will structure and connect this content. The first step of this is to find what the users believe should be organized together, and what they consider different elements. A fitting method for such an investigation is the \textit{card sorting technique.} In her book exploring this method Donna Spencer defines card sorting as \textit{a tool that helps us understand the people we are designing for} \parencite[][6]{WEB:DonnaSpencer}..... \fxnote{Nåede lige ikke mere}



\section{Fitting the methods to the task}
\label{FittingTheMethods}





































