\chapter{User Study of The TP Community}
\label{ChapterWorkshop}
The tasks presented in \autoref{TaskAndMethods} serve as proposals for how to include the users as an active part of the design process at TC Electronic in correlation with developing a platform for sharing User TonePrints. The tasks are all considered equally important, but ultimately only one of them is explored in this project due to the restricted timeframe. Through discussions with our company supervisor, task 1 is chosen for this purpose. The following chapter will start with an elaboration on this decision before moving on to the execution and subsequent analysis of the results, including what methods are applied for both parts.

\section{Deciding on a task}
\label{TaskDecision}
As described in \autoref{Task1}, the purpose of task 1 is to investigate how to design the information architecture of the TonePrint community, in order to ensure that the design accommodates to the users' mental model of the system. Many considerations were put in to the decision of conducting this task, including TC's wishes, how well it would fit into a typical SCRUM sprint, the remaining timeframe for this project etc. The impact of all these factors also influenced what methods to consider and how to approach them.

Firstly, an investigation of the information architecture seemed the most fitting to conduct due to the current state of the TonePrint community. TC are at a very early stage of development, as no version of a sharing platform currently exist. TC therefore considered it evident to start with an assessment of how content and features should be structured as a baseline for the future design process.

Secondly, previous investigations in this project have showed how TC follows the SCRUM framework in the shape of 3-week sprints (see \autoref{ThemanticAnalysis}). As such, much of the discussion was on finding a task and suitable method that could easily fit to this timeframe, and task 1 was here considered ideal. the limited time remaining on the project furthermore required to conduct it as a 2-week sprint, and even though this is not the same duration as a sprint at TC, conducting a task in this timeframe will still provide insight into how UX tests can fit to a short time period. Task 3 and 4 could just as well fit this timeframe, but each of them require an existing app to some extend. It wouldn't necessarily have to be a fully functional system, but still one that could represent the appearance and functionalities of the system.

Thirdly, task 5 focusing on creating personas of the typical users of TC products was considered very valuable for TC in their understanding of them. However, this task is too extensive and time-consuming to execute in a 2-week sprint and is in general considered a semester project on its own. If it were to be executed anyway, it would only result in demographical info on the users, which doesn't provide anything new for TC.\\

\noindent
When choosing task 1, further specifying the approach is required, as it proposes two different ones. One of these is through Lo-fi testing which is a great tool, but as it is elaborated on in \autoref{Task1}, it requires a rough idea of how the system should work and how the information should be presented. For the TonePrint community there currently is no rough idea of how it should look, and it would as such make more sense to include the users in shaping the concept and flow of the community from the bottom through a \textit{participatory design procedure}. The approach should then follow the outlined suggestion in \autoref{Task1} by conducting a workshop with potential end-users as subjects in order to explore their mental models of a TonePrint community.

\section{Mental Models}
\label{MentalModel}
Before proposing the experimental design and setup of a workshop, it seems appropriate to first define what is meant by \textit{mental models}, and how they are constructed. In \autoref{UXbenefits} it is mentioned that if the developers have the same understanding of how a given product should look and behave, then the users are more likely to have a pleasant experience. Other factors such as the users' hedonic opinions of the product also play a role, but this was also meant as a quick introduction to the term. In general, mental models are types of conceptual models that resides in people minds, as the name implies. In other words, mental models are the users' understanding of how something works. The definition of a conceptual model is already outlined in \autoref{CommunityConcept} as an explanation of how something works, including all concepts in it, and how they relate to each other. For mental models, different people may hold different versions for the same system, and some people may even hold multiple models of the same system, focusing on different aspects of its operation \parencite[][26]{PDF:DonNorman}. \\

\noindent
When it comes to constructing mental models, \textcite{PDF:MentalModelInstructions} defines three different factors that influence this process. These are the individuals ability to utilise existing models that relate to the matter in question, general observations of the outside world, and explanations from other people \parencite[][68-69]{PDF:MentalModelInstructions}. It should be noted that the "existing models" can be both internal and external models of similar systems, as long as they can serve as analogies for the model to be created. When this is put into context with the TonePrint community, it is as such expected that the users will construct mental models on the basis of:
%
\begin{itemize}
	\item Their understanding of how the current TonePrint app works, including TC's influence on this understanding.
	\item Observations of how other users utilise the TonePrint app and currently attempt to share User TonePrints with each other.
	\item What other end-users i.e. guitarists tell them.
\end{itemize}
%



%\parencite{WEB:NielsenNormanCardSort}\\
%\parencite{WEB:ConceptMapAnalysis}\\

\section{Experimental design}
\label{ExperimentalDesign}
For task 1, it's important to remember that the main goal is to investigate possible end-users' mental models of a sharing platform for User TonePrints, in order to asses how the information architecture of it should be shaped. The employment of end-users and not generic users is important, as they represent the mental models of the people who are most likely to use a finished version of the system. The current TonePrint app already aims at a certain audience, i.e. guitarists, and employing subjects with no experience with this field wouldn't result in an appropriate outcome of the study.

Shaping the information architecture of a system requires multiple steps of generating ideas for content and communicating these, before moving on to testing them with the users \parencite[][356-364]{PDF:InformationArchitecture}. For this task, the first step would as such be to generate ideas for content and features to be included, and how they should communicate with each other? Instead of engaging in a phase of generating such ideas, this paper instead suggests to make use of studies already conducted by TC focusing on a sharing platform for User TonePrints. A study by our company supervisor \textcite{PDF:BrugerWorkshopUserTonePrints}, investigates what is required from such a platform, and the result is given in the shape of a list of content which, if included, would make end-users more interested in using the platform \parencite[][35]{PDF:BrugerWorkshopUserTonePrints}. It's important to note that this content also is produced by the users in a workshop, and by utilizing this information, it will serve as a good starting point. Furthermore, the interview conducted with members of the TC staff previously in this project may also contain appropriate content for developing an appropriate strategy for the information architecture.

\subsection{The card sorting method}
\label{CardSort}
As the previously mentioned sources of inspiration provide meaningful content for a sharing platform for User TonePrints, the next step is then to investigate how the users will organise this content. A suitable approach for this purpose is by utilizing the card sorting method. Donna Spencer describes this in her book as a fairly straightforward tool that helps the developers understand the people they are designing for \parencite[][6]{WEB:DonnaSpencer}. The method involves users sorting a set of cards into piles of what they find similar either according to predetermined group names, or they can be told to describe the categories themselves after the sort. These are referred to as an \textit{open card sort} and a \textit{closed card sort} respectively.

Each card represents an item formulated either from the study by TC Electronic or from the interview previously in this project, and while making these, some considerations must be made to the number of cards formulated. 



%Argument for remote card sort: TC's bruger er spredt ud over hele verden. Ved at sende det ud via de rigtige medier, kan vi få mere en mere spredt demografi i stedet for bare at bruge dem, der nu er til at få fat i i nærheden af os.
		%I såfald skal det også nævnes, at vi lokker dem med en konkurrence om en pedal, som TC selv er gået med til at udlodde.

%Selvom en card sort er et stærkt værktøj, der kan danne et billede af, hvordan information skal organiseres (hvad skal placeres sammen og hver for sig), så kan vi ikke kun ud fra en card sort tegne en informationsarkitektur. Vi er også nødt til at vide, hvordan disse grupperinger snakker sammen.

%Husk at inkluder consent form til workshoppen i appendix

\noindent
For the purpose of using the users mental models to shape the IA of the TonePrint Community, it's desired to look at similarities of the mental models. \textcite{WEB:ConceptMapAnalysis} suggest a method for analyzing a teams Shared Mental Model (SMM). SMM represents a mental model shared by a group of people working together i.e  at solving a problem or a task. It's important that there is a some what mutual consensus of the SMM, otherwise it would be difficult for the people working together, because they may not know the purpose of what their fellow team members do and it could cause miss communication. The scope of identifying the SMM in this study is however not to evaluate how well users may work together. It's desired to investigate this SMM let the IA accomodate the different mental models \autoref{TaskDecision}.


\section{Analysis Constructed Shared Mental Model}
\label{ACSMM}
To analyze the SMM it's proposed by \textcite{WEB:ConceptMapAnalysis} to create a Analysis Constructed Shared Mental Model (ACSMM). \autoref{fig:ACSMM} illustrates a very simplistic view of the process of constructing a ACSMM. At the beginning there is the users Individual Mental Model (IMM). These IMM is what is addressed in \autoref{MentalModel}. The next part is to acquire the users Individual Constructed Mental Model ICMM, which is their mental model putted in their own words. In this workshop the ICMM's is visualized be the concept maps created by each subject, as described in \autoref{WorkshopApproach}. The ICMM's is then analyzed and compared, so that a ASCMM may be created for explaining a common mental model shared by the users \autoref{fig:ACSMM}.  

\begin{figure}[H]
	\centering
	\includegraphics[width=\textwidth]{ACSMM.pdf}
	\caption{Illustration of the process of crating a ACSMM}
	\label{fig:ACSMM}
\end{figure}

The step of going from multiple ICMM to a ACSMM is described in \textcite{WEB:ConceptMapAnalysis} as a ICMM coding phase, a shared analysis phase and a SCSMM construction phase. In the coding phase the concepts, links and cluster depict in the ICMM's are transformed into codes, that is comparable. The coding process for this workshop will not only be conducted for the subjects concept maps, however will it be based on the subjects presentation of their concept maps, which has been recorded \autoref{WorkshopApproach}. This is due to the concept maps being constructed quite different and because when putted to word, a more representative description of the IMM may be derived. In the shared analysis phase the codes from each subject is analyzed in order to determine which items of the ICMM's the subjects shares. To determine this a criterion is set for percentages of subjects sharing an item. \textcite{WEB:ConceptMapAnalysis} suggest 50\% as criterion. Finally is the ACSMM constructed by taking the shared items from the previous step and use them to construct a Concept Map. After this the shared items is used to assembly a model which serves as the ACSMM. 










%\subsection{Shared Analysis}
%\label{SharedAnalysis}
%%
%The next phase is the shared analysis in which the codes derived from the ICMM's is used, in order to find items (concept and links) which is shared between them. In order to determine which items that identifies as shared a criterion needs to be defined. As described in \autoref{ACSMM} it's suggested to set the criterion at 50\% at the beginning, which afterwards may be adjusted. This means that a given item have to occur in atleast in 50\% of the ICMM's to be identified as shared.\\
%To compare the codes they are first typed into tables, for the purpose of gathering the codes for easier overview. The tables containing the codes is \autoref{tab:Subject1Coded} to \autoref{tab:Subject5Coded}. 
%
%The lists containing the subjects individual codes, \autoref{tab:Subject1Coded} to \autoref{tab:Subject5Coded} are compared to each other to find the concepts which is shared. As recommended by \textcite{WEB:ConceptMapAnalysis} is the criterion for a concept to be identified as shared defined as 50\%. The concepts which is defined as shared is listed at \autoref{tab:SharedConcept}(WHich isn't finish). 
%
%It's deemed problematic to use the described steps from \textcite{WEB:ConceptMapAnalysis} to create a ACSMM. This is because the task of the workshop was for the subjects to solve a task with a they have to build up as they go. This gives the users full control in regards to which features that exists in the system, how they works and when they wants to use them. This results in the subjects going different ways around the task and solving it in their won way. This is quite similar to some system in the real world which give you the option to reach a certain goal different ways. This however creates one big problem regarding, them only describing the part of the system, which they uses. This makes it very difficult to compare one persons ICMM with another, because they may have chosen different ways of solving the task and theirby having a minimal of similar concepts in common. This doesn't necessarily mean that the two subject disagree on how the system should work and their mental models might be similar towards the system. They just have chosen different approaches for the task, ehich means that they don't uses the same branches of the system. \\
%
%\begin{table}[]
%	\centering
%	\begin{tabular}[width=\textwidth]{c|lllllc}
%\hline 
%Shared item & subject 1 & subject 2 & subject 3 & subject 4 & subject 5 & \%	\\ \hline
%TP App & TP App & App & TP App & TP App & TP App & 100\%	\\ 
%My User Toneptints & My TonePrint site &  &  & User TonePrints \& Own User TonePrint Library & Personal Library & 60\% \\
%Search function & Search Friends & Search in List & & Search User Name \& Search Function & Search, Advanced search \& Meta Search Option & 80\% \\
%User 1 Profile & User 1 Profile & & User 1 Profile & User 1 Profile site & & 60\% \\
%User 1 TonePrints & User 1 TonePrint list & & User 1's Toneprints & User 1 TonePrints
%	\end{tabular}
%\end{table}
%
%\begin{table}[H]
%	\centering
%	\begin{tabular}[width=\textwidth]{|l|c|}
%	\hline
%	Shared concept & Sharedness level (\%) \\ \hline
%	TP App & 100\% \\ 
%	User TonePrint Library & 60\% \\ \hline
%	\end{tabular}
%	\caption{Shared concepts and sharedness level}
%	\label{tab:SharedConcept}
%\end{table}





















