\chapter{User Study of The TP Community}
\label{ChapterWorkshop}
The tasks presented in \autoref{TaskAndMethods} serve as proposals for how to include the users as an active part of the design process at TC Electronic in correlation with developing a platform for sharing User TonePrints. The tasks are all considered equally important, but ultimately only one of them is explored in this project due to the restricted timeframe. Through discussions with our company supervisor, task 1 is chosen for this purpose. The following chapter will start with an elaboration on this decision before moving on to the execution and subsequent analysis of the results, including what methods are applied for both parts.

\section{Deciding on a task}
\label{TaskDecision}
As described in \autoref{Task1}, the purpose of task 1 is to investigate how to design the information architecture of the TonePrint community, in order to ensure that the design accommodates the the users' mental model of the system. Many considerations were put in to the decision of conducting this task, including TC's wishes, how well it would fit into a typical SCRUM sprint, the remaining timeframe for this project etc. The impact of all these factors also influenced what methods to consider and how to approach them.

Firstly, an investigation of the information architecture seemed the most fitting to conduct due to the current state of the TonePrint community. TC are at a very early stage of development, as no version of a sharing platform currently exist. TC therefore considered it evident to start with an assessment of how content and features should be structured as a baseline for the future design process.

Secondly, previous investigations in this project have showed how TC follows the SCRUM framework in the shape of 3-week sprints (see \autoref{ThemanticAnalysis}). As such, much of the discussion was on finding a task and suitable method that could easily fit to this timeframe, and task 1 was here considered ideal. the limited time remaining on the project furthermore required to conduct it as a 2-week sprint, and even though this is not same duration as a sprint at TC, conducting a task in this timeframe would still provide TC insight into how UX tests can fit to a short time period. Task 3 and 4 could just as well fit this timeframe, but each of them required an existing app to some extend. It wouldn't necessarily have to be a fully functional system, but still one that could represent the appearance and functionalities of the system.

Thirdly, creating personas of the typical users of TC products was considered very valuable for TC in their understanding of them. However, this task was too extensive and time-consuming to execute in a 2-week sprint and is in general considered a semester project on its own. If it were to be executed anyway, it would only result in demographical info on the users, which isn't either very valuable for TC.\\

\noindent
When choosing task 1, further specifying the approach is required, as it proposes two different ones. One of these is through Lo-fi testing which is a great tool, but as it is elaborated on in \autoref{Task1}, it requires a rough idea of how the system should work and how the information should be presented. For the TonePrint community there currently is no rough idea of how it should look, and it would as such make more sense to include the users in defining and shaping the concept and flow of the community through a \textit{participatory design procedure}. The approach should then follow the outlined suggestion in \autoref{Task1} by conducting a workshop with potential end-users as participants in order to explore their mental models of a TonePrint community.


\section{Mental Models}
\label{MentalModel}


\section{Workshop approach}
\label{WorkshopApproach}












