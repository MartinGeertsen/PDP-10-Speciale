\chapter{Thematic Analysis}
\label{ThemanticAnalysis}

\section{Method}
\label{ThematicMethod}

\begin{itemize}
	\item Til et semistrukturet interview er det svært at følge en prædifeneret analyse, da man ikke ved hvilken vej interviewet tager.
	\item Vi valgte den tematiske analyse fordi man kunne komme godt ned i ens data og laver et overblik, før man analyserede.
	\item Beskriv stepne fra kilden.
\end{itemize}

\section{Themes}
\label{ThematicThemes}
As described in (Braun and Clark (lav kilde)) does the thematic analysis create a understanding of the interview data by thoroughly coding the transcribed interview data, whereafter the codes are used to create themes, that can be used to interpret the interview. \\

The four interviews were given a total of 272 codes, from which several codes did cover more than one interesting aspect and is hence present in more than one theme. As result of an iterative process of dividing the codes into themes, a total of 35 themes were created. Some of the themes are strongly connected by addressing some of the same areas, but are divided to create more specific themes rater than to general. The themes are in danish and is shown in \autoref{ThemesOverview}. This is followed by a description of the theme.


%\newcolumntype{R}{>{\raggedright\arraybackslash}X} %Left adjust and no stretch of text. Good for narrow cells
%\begin{table}[H]
%\label{ThemesOverview}
%\small
%\begin{tabularx}{\textwidth}{|R|R|R|R|R|}
%\hline
%Beaming App & Ikke inspireret af andre & Udviklingsværktøj & Parameterdesign & Rollefordeling / Hieraki \\ \hline
%Prioritering af features & Beslutning på baggrund af test & Erfaring fra tidligere produkter & Målrettet mod brugergrupper & Beslutning på baggrund af antagelser \\ \hline
%Beslutning på baggrund af bekvemmelighed & Kommunikation i udvikling af app & Forretningsmodel & marketing \/ TonePrint beskrivelse & TonePrint koncept \\ \hline
% Parametre UI & Ingen brugerinddragelse & Beslutning på baggrund af personlige holdninger & Inspiration af eksterne Produkter & Rod \\ \hline
%UI design & Brugerinddragelse & Eksterne udviklere & Brugerfeedback & Brugerinddragelse \\ \hline
%Community efterspørgelse & Community tags & Tilbageholdt deling & Community beslutning & Feedback om langsom app \\ \hline
%Fokus på brugere og brugervenlighed & Prioritering af TonePrint appen & Samarbejde med kunstnere & Tekniske begrænsninger & Ingen målsætninger i forhold til TonePrint appen \\ \hline
%\end{tabularx}
%\caption{Themes overview}
%\end{table}

%%%%%
%Alt herfra indtil conclusion skal i appendix
%%%%%


\section{Interview conclusions}
\label{InterviewConclusion}
Som resultat af analysen kan det ses at SCRUM er en meget vigitg del af mentaliteten hos TC og at den måde at opsætte krav for sprints og prioritere features, har ens stor betydning for deres udviklings process. Det virker til at den erfaring de har fra tidligere brugerinddragelse har være meget god, dog ikke den ene gang med Jesper, hvor det virker til at timingen har været forkert. Det virker til at de har en idé om hvem deres brugere er, dog uden helt at vide det, samtidig med at de ikke rigtig har erfaring med at målrette efter bestemte brugere når de designer, da de mener at TonePrint konseptet aspirere nok til deres brugere. Det ses at de har en masse idéer til TonePrint communitiet, hvor de fleste er enige om at det med Tags, er en vigtig del at få undersøgt og lavet.
