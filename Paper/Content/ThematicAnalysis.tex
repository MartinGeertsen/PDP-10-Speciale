\chapter{Thematic Analysis}
\label{ThemanticAnalysis}

\section{Method}
\label{ThematicMethod}

\begin{itemize}
	\item Til et semistrukturet interview er det svært at følge en prædifeneret analyse, da man ikke ved hvilken vej interviewet tager.
	\item Vi valgte den tematiske analyse fordi man kunne komme godt ned i ens data og laver et overblik, før man analyserede.
	\item Beskriv stepne fra kilden.
\end{itemize}

\section{Themes}
\label{ThematicThemes}
As described in (Braun and Clark (lav kilde)) does the thematic analysis create a understanding of the interview data by thoroughly coding the transcribed interview data, whereafter the codes are used to create themes, that can be used to interpret the interview. \\

The four interviews were given a total of 272 codes, from which several codes did cover more than one interesting aspect and is hence present in more than one theme. As result of an iterative process of dividing the codes into themes, a total of 35 themes were created. Some of the themes are strongly connected by addressing some of the same areas, but are divided to create more specific themes rater than to general. The themes are in danish and is shown in \autoref{ThemesOverview}. This is followed by a description of the theme.


\newcolumntype{R}{>{\raggedright\arraybackslash}X} %Left adjust and no stretch of text. Good for narrow cells
\begin{table}[H]
\label{ThemesOverview}
\small
\begin{tabularx}{\textwidth}{|R|R|R|R|R|}
\hline
Beaming App & Ikke inspireret af andre & Udviklingsværktøj & Parameterdesign & Rollefordeling / Hieraki \\ \hline
Prioritering af features & Beslutning på baggrund af test & Erfaring fra tidligere produkter & Målrettet mod brugergrupper & Beslutning på baggrund af antagelser \\ \hline
Beslutning på baggrund af bekvemmelighed & Kommunikation i udvikling af app & Forretningsmodel & marketing \/ TonePrint beskrivelse & TonePrint koncept \\ \hline
 Parametre UI & Ingen brugerinddragelse & Beslutning på baggrund af personlige holdninger & Inspiration af eksterne Produkter & Rod \\ \hline
UI design & Brugerinddragelse & Eksterne udviklere & Brugerfeedback & Brugerinddragelse \\ \hline
Community efterspørgelse & Community tags & Tilbageholdt deling & Community beslutning & Feedback om langsom app \\ \hline
Fokus på brugere og brugervenlighed & Prioritering af TonePrint appen & Samarbejde med kunstnere & Tekniske begrænsninger & Ingen målsætninger i forhold til TonePrint appen \\ \hline
\end{tabularx}
\caption{Themes overview}
\end{table}

\subsection*{Beaming App}
\label{ThemeBeamingApp}
This theme describes that TC Electronic earlier had a TonePrint app which did not include a TonePrint editor part, that allowed the creation of User TonePrints. The app consisted of a library with Artist TonePrints that could be filtered for guitar or bass, and with it the options to beam the TonePrint through the instruments pickups to the pedal.

\subsection*{No inspiration from others (Ikke inspireret af andre)}
\label{ThemeNoInspirationFromOthers}
This is a small theme containing comments that TC Electronic hasn't taken inspiration from other products, neither external nor internal, for the development of the TonePrint app. However is on commenting that on of the sister companies has a product with some similarity, but he don't think that it has worked as inspiration.

\subsection*{Development tool (Udviklingsværktøj)}
\label{ThemeDevelopmentTool}
The TonePrint concept and the editor is a evolution of system that have been a part of TC Electronics for a long time. The system Virtuel Front was a system that TC Electronic used to create new pedals, by determine the values of different parameters of the product and determine which parameters the users could alter with the physical knobs. This system was not anything that TC offered to their costumers. This system was very complicated, and was mostly used by the audio engineers at TC Electronic. In the beginning of the development the TonePrint concept, Virtuel Front was used to set the parameters of the TonePrints and has since been developed to become the current editor. 

\subsection*{Parameter Design (Parameterdesign)}
\label{ThemeParameterDesign}
Given the way that TC Electronics products work, is the different parameters of the audio settings set by models, described as Meta Models. This models is what is altered to define the sound of the products. A problem at TC Electronic has been to make the models readable for the users e.g. how gain is controlled, with labels, intervals etc. These models is what the editor is altering and that have given a problem, because TC Electronics still have to hide some of the models, to avoid showing all of their concept and risk being copied. 

\subsection*{Roll allocation / hierarchic (Rollefordeling / Hieraki)}
\label{ThemeRollAllocation}
At some point in the development process, desisions has to be made, and it's commented that the product manager has the final saying. This has lead to scenarios where the programmer doesn't even know how to describe a feature because he think it's very complicated and didn't make sense to him, however has it been implemented because the decision was taken bu others. In another scenario however, did the designer insist that a feature had to be done a certain even though he was told it would be to difficult. But through dialog and stubbornness, was the feature created as the designer wanted.

\subsection*{Prioritizing of Features (Prioritering af features}
\label{ThemePrioritizingOfFeatures}
When TC Electronics are developing new products they of cause have to plan out which features to include. Every employee can come with suggestions, and some often doe. Here they also try to listen to their uses, because when they get features recuests from their users on e.g. Facebook, is the idea stored together with the ideas from the employees. When it comes to implementing the features the decision is made upon how easy and quickly an feature is to implement, and how important the feature is. Hence are some features implemented even though it isn't very important, because it's easy and quick to implement. This does however also work the other way around whereas the community has been a feature request, almost since the beginning of the TonePrint concept. The problem has been that it was deemed to difficult and time consuming to create at the beginning, so the idea was putted away, so other easier features and task could be done.

\subsection*{Decisions based on test (Beslutninger på baggrund af test)}
 \label{ThemeDecisionsBasedOnTest}
 TC Electronic has conducted few user studies which have led to decisions in the development process. One time they wanted to make a certain type of pedal, on a market they weren't currently in. There fore they made a user test on a competitors product, which indicated that the interface and controls of the pedal were to complicated. This knowledge was used to design their own pedal, which became a success. The concept of making the editor available for users, instead of just using it as a development tool, also came from a user study. Here a group a bass players were given access to the Virtuel Front \autoref{ThemeDevelopmentTool}, with the task of creating a new sound for a bass amp, which resulted in a new setting for the TC products. This opened the eyes for TC Electronic for giving the users this opportunity, which became the TonePrint editor. Finally has there been conducted a user workshop for developing ideas for the TonePrint Community(Jespers rapport). This workshop has led to several ideas for which features that should be included in the community. One of the decisions is that the users shouldn't be constrained by categorize created by TC Electronic, which has led to the idea of 'tags', which has it's own theme.
 
 \subsection*{Experience from former products(Erfaringer fra tidligere produkter)}
 \label{ThemeExperienceFromFormerProducts}
 This theme highlights that TC Electronic uses their prior experience which they have obtained through other products. This includes a pedal that was to complicated to use and to expensive, which resulted in a failed product. Another experience is the pedal mentioned in \autoref{ThemeDecisionsBasedOnTest}, where a user test of a competitors product led to the success of their own. They have also drawn on experience when creating the TonePrint concept, because they earlier have experience that when creating complex systems, they have to make it more simple to ease the use of the product.
 
\subsection*{Focused on target group(Målrettet mod brugergrupper)}
\label{ThemeFocusedOnTargetGroup}
There are several target user groups for both the TonePrint concept in general, and the TonePrint Community. With the TonePrint concept there are users whom just wants a regular pedal, with which you can control the settings with the physical knobs. Then there are the users whom have a idol they wants to sound like, or just love to discover new sounds made by professionals, which is accommodated by the Artist TonePrints and templates. The third group are the "tweakers" whom like to go in depht with the different parameters and create their own TonePrints. These target groups are not anything that TC has found by doing any investigations, but was more like a gut feeling. For the community is there also a target group which are users whom wants to share their TonePrints, so that other people can use it.

\subsection*{Decisions based on assumptions(Beslutning på baggrund af antagelser)} 
\label{ThemeDecisionsBasedOnAssumptions}
This theme highlights that TC Electronic doesn't have much experience with including user studies or investigating their users. The decisions of which features to include and how to design them is mostly based on assumptions of what the users wants and what works best. Fore example is the interface for the parameter settings design, because it was mend to be the most natural way to design it. In total is it clear that they base their development on assumptions of the users need, abilities and wishes.

\subsection*{Decision based on conviviality(Beslutning på baggrund af bekvemlighed)}
\label{DescisionBasedOnConviviality}
Here to scenarios are mentioned where conviviality have played a role in how a decision was made. Firstly is the idea of implementing links to to youtupe and the likes, when creating a TonePrint for the community. This is something they are almost sure they will include, because they already have the technology to implement it. The second scenario is that when they design new solutions have the orpotunity to test different ideas, if the implementation is easy enough, otherwise wouldn't they test the ideas first.

\subsection*{Communication in the development of an app (Kommunikation i udviklingen af en app)}
\label{ThemeCommunicationInTheDevelopmentOfAnApp}
It seams like there hasn't been a perfect communication between all members of the development team, when creating the TonePrint App. It's found that the differences that are depending on which platform that is used, stems from individual programmers for each platform, whom have used their 'artistic freedom' and are allocated on different teams. However some of the difference is depending on what is commonly used at the specific platform. Another example is that the designer of the parameters interface didn't know how they worked when he designed them, so he designed a interface for something he didn't understand, which he felt was problematic.

\subsection*{Business model (Forretningsmodel)}
\label{ThemeBusinessModel}
The TonePrint concept enables TC Electronic to always upgrade their product, by adding new content, in the form of new artists and templates. In this theme it's also highlighted that TC is a business, whom have to separate from their otherwise competitors, which also i way they made a user test of competing product, \autoref{ThemeDecisionsBasedOnTest}.

\subsection*{Marketing / TonePrint description(Marketing / TonePrintbeskrivelse)} 
\label{ThemeMarketingTonePrintDescription}
For every TonePrint in the library is there a text description which tells something about the artist and the specific TonePrint. The ideas to what is ridden in thous descriptions normally stems from the crew whom are with the artist under the description of the TonePrint. But in any cases is the marketing team going through it to ensure it fits its purpose. The aim with these descriptions is to engage the users and to make them want to try out the TonePrint. Also does it help give some information about the purpose of the settings in the TonePrint. The description part is also important when looking at the TonePrint Community, because there could easily come alot of TonePrints, where these descriptions can help the users sort out which ones they want to try. 

\subsection*{TonePrint Concept}
\label{ThemeTonePrintConcept}
The TonePrint concept have three layers which are, regular pedal adjustment, beaming of Artist TonePrints and templates and finally the creation of User TonePrints. Each of these layers aspires to different target groups \autoref{ThemeFocusedOnTargetGroup}.\footnote{Tag den her senere}.

\subsection*{Parameter UI}
\label{ThemeParameterUI}
When going from the former TonePrint editor to the current alot of the UI has been changed. Earlier every interaction has been controlled with sliders, and there were a problem with mapping the sliders and how they affected the parameter. Another problem with the old design was that the sliders took up alot of space, which would be very messy for a phone size interface. So the focus of the new design was also to make it more user friendly.\\
Currently a problem that is discussed is the lack of information at for each parameter. There is no information of what a parameter changes, so the only way to figure this out is to try new settings and listen. To accommodate this problem are they discussing groupings and descriptions, which in the end should enable the user to better understand how to interact with the sliders to reach a desired effect.\footnote{Anything IMPORTANT missing}

\subsection*{No user involvement (Ingen brugerinddragelse)} 
\label{ThemeNoUserInvolvement}
This is a very small theme which only highlights that TC Electronic hasn't used much user involvement and decisions are made without including the users. 

\subsection*{Decisions based on personal opinions (Beslutning på baggrand af personlige holdninger)} 
\label{ThemeDecisionsBasedOnPersonalOrpinion}
Much of TC Electronics development is based on personal opinions in the development teams. This theme shades further light on the fact that there is little communication with the users of the systems. The decisions on how to design the informational structure and how to design the TonePrint app, was all based on gut feeling. Gut feeling seams to be a important part of how decisions are taken at TC Electronic.

\subsection*{Inspiration from external products (Inspiration af eksterne produkter)} 
\label{ThemeInspirationFromExternalProducts}
In this theme it's become clear that TC Electronic have been looking at other companies for inspiration for the TonePrint Community. They are looking at Yamahas Soundmondo and Fyres Effects, which both have some sort of community. From their search have they discovered that having a 'like' system, can result in many effects not being presented, because they haven't been rated yet. When looking at others solutions, they find both inspiration, and what works and what doesn't.

\subsection*{Mess (Rod)} 
\label{ThemeMess}
This theme contain codes which aren't found useful and will not be described further.

\subsection*{UI Design} 
\label{ThemeUIDesign}
\footnote{Den her forvirer mig}

\subsection{Native UI}
\label{ThemeNativeUI}
\footnote{Den her står ikke på listen og den børe måske stå sammen med Langsom app Feedback}

\subsection*{User involvement (Brugerinddragelse)} 
\label{ThemeUserInvolvement}
As seen in \autoref{ThemeDecisionsBasedOnTest} isn't user involvement completely new to TC Electronic. The development of the TonePrint Editor was based on a user study, where bass players had the opportunity to create a new effect for a bass amp. Another example is the user focus group workshop described in (JESPERS KILDE), which purpose was to shad light on what users wold like with a TonePrint Community. A repeating thing in the this theme is that they want to give the users what they wants and involving the users, are to figure out what they want the most.


\subsection*{External developers (Eksterne udviklere)} 
\label{ThemeExternalDevelopers}
TC Electronic have earlier used designers and developers, whom haven't been a employee to create a product. The first TonePrint app was designed by an external designer, but implemented by TCs own programmer. 

\subsection*{User feedback (Brugerfeedback)} 
\label{ThemeUserFeedback}
There are typically to ways that TC receives feedback from their users, through their facebook site 'TonePrint Junkies' and their Music Tribe Community Forum. They receive a lot of feedback regarding feature requests, which they are noting down, so that it might be looked at, in their next development phase. The feedback is also related to problems regarding their products. Every member of the development team is a member of the facebook site, so that they have some user contact, but most of the feedback they are receiving has been through the marketing team, whom selects whats needed to be resolved.

\subsection*{Community demand (Community efterspørgsel)} 
\label{ThemeCommunityDemand}
From the beginning of the TonePrint concept has there been a request from the users, to enable a way of sharing TonePrints with each other. TC Electronic has acknowledged this request for a while, but as mentioned in \autoref{ThemePrioritizingOfFeatures} has it been a to big a task for them earlier. Now they are beginning to start the development of this community, because they see it as a matter of time, before the users finds another solution.

\subsection*{Detain sharing (Tilbageholdt deling)} 
\label{ThemeDetainSharing}
Since the users started requesting a TonePrint Community have TC Electronic locked the files which forms the TonePrint. This has been done because they wanted to bee in charge of the sharing of the TonePrints. This has led to some users making creative workarounds, where they have shared pictures of their settings, so that they may be copied.

\subsection*{Community tags} 
\label{ThemeCommunityTags}
One feature that seams to be a major interest for the TonePrint Community, is a feature referred to, as 'tags'. The scope is that a users should put some tags on a new TonePrint he or she just have created, which should make it easier to find for the other users. Some suggest that the users should be completely free to create their own tags, without limitations. Others suggest that there should be an number of options to chose from. They are however aware that there could be a problem because tags can be misleading, given they may describe something that is subjective. At TC Electronic this is seen as a very important feature to get right.

\subsection{Community Features 'Not tags'}
\label{ThemeCommuninityFeaturesNotTags}
Besides the feature of tagging TonePrints is there a general idea of creating a follow or subscribe feature. This would enable users to follow other users and be updated when he or she creates new TonePrints. This may generate motivation to be more active on the community, and give the users the opportunity to become idols them selves. Being able to make descriptions of the TonePrints and linking to soundcloude or youtube is also a feature they think will help motivating the use of the community. Most of the ideas for the community relies on the users having an account, which currently isn't a part of the TonePRint concept, which means they also have to create data bases for this. \footnote{Den her bør vi kigge mere på, da de features der er her i kan sættes op som en model for communitiet}

\subsection*{Community decision (Community beslutning)} 
\label{ThemeCommunityDecision}
This is a very broad theme where different decisions and ideas for the community in general. One decision which they don't fully agree on yet is how and how much the community should be integrated in the current TonePrint app. Some argue that having it as a fully integrated system with the community being a option, on the same level as the editor part in the current version, will make the system easier to use. Others argue that it should be divided, so that a few of the sharing/uploading thing are in the app, but the rest of the "social media" features, are a thing on its own. \\
A decision they seams to agree upon is that they want to involve the users in the development of the community, to ensure that they meet the users wishes and expectations. 

\subsection*{Slow app feedback (Feedback om langsom app)} 
\label{ThemeSlowAppFeedback}
After releasing the newest TonePrint app did TC Electronic receive some negative feedback. Many users had a problem with the app being so slow that it was almost useless. This was a result trying to make a unified solution for both android and iphone. TC took the problem very seriously, and repaired the problem, be making separated UI solutions. 

\subsection*{Focus on users and user friendliness (Fokus på brugere og brugervenlighed)} 
\label{ThemeFocusOnUsersAndUserFriendliness}
This theme underline that TC Electronic have some knowledge of their users and that they wants their users to easily be able to use their products, even though they haven't included them much in their development so fare. There are some comments mentioning that the users and user friendliness are a focus, however doesn't it seam like there are a understanding of how and why.

\subsection*{Prioritizing of the TonePrint app (Prioritering af TonePrint appen)} 
\label{ThemePrioritizingOfTheTonePrintApp}
The TonePrint app has become a product prioritized on the same level as the other products at TC Electronics, even though it's a free product. This has resulted in more resources has been allocated to the development. A result of this has also been that the development of the community at one point has been pushed further away.

\subsection*{SCRUM}
\label{ThemeScrum}
It's very clear from this theme that the agile development method SCRUM plays a major role in the development at TC Electronics. Each sprint is three weeks, and they are quiet strict on following the sprint rhythm. Up to a sprint they check the backlog tp ensure they have the right thing of interest, and it's in the backlog new feature request ends up, if they are deemed necessary to implement. If a task seams to take longer than that of the deadline, it's deconstructed into smaller tasks, which than is transferred to the next sprint log.\\
There is an overall agreement that the SCRUM method are beneficail for the development team. They seams to agree that it helps focus on the right task at the given time, because they dond't have to think on the tasks, which are planed for the following sprint. \footnote{Man kan måske gå mere i dybten, men det vil jeg hellere i Resultat/interview konklusionen} 

\subsection*{Collaboration with artists (Samarbejde med kunstnere)} 
\label{ThemeCollaborationWithArtists}
\footnote{Slet?}

\subsection*{Technical limitations (Tekniske begrænsninger)} 
\label{ThemeTechnicalLimitations}
\footnote{Slet?}

\subsection*{No objectives in relation to the TonePrint app (Ingen målsætninger i forhold til TonePrint appen)} 
\label{ThemeNoObjectivesInRelationToTheTonePrintApp}
It seams like there aren't any milestones for the TonePrint app, neither in terms of downloads, use ore reputation. 

\section{Interview conclusions}
\label{InterviewConclusion}
Som resultat af analysen kan det ses at SCRUM er en meget vigitg del af mentaliteten hos TC og at den måde at opsætte krav for sprints og prioritere features, har ens stor betydning for deres udviklings process. Det virker til at den erfaring de har fra tidligere brugerinddragelse har være meget god, dog ikke den ene gang med Jesper, hvor det virker til at timingen har været forkert. Det virker til at de har en idé om hvem deres brugere er, dog uden helt at vide det, samtidig med at de ikke rigtig har erfaring med at målrette efter bestemte brugere når de designer, da de mener at TonePrint konseptet aspirere nok til deres brugere. Det ses at de har en masse idéer til TonePrint communitiet, hvor de fleste er enige om at det med Tags, er en vigtig del at få undersøgt og lavet.
