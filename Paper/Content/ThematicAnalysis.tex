\chapter{Results of thematic analysis}
\label{ThemanticAnalysis}

%\begin{itemize}
%\item Themes
%		\begin{itemize}
%			\item Present a overview of the themes
%			\item Link to appendix with a description of each theme
%			\item Describe the overall finding from the interview, with references to the Themes, which should refer to timestamps in the transcription.
%			\item (It’s important we highlight something related to SCRUM, decisions making, Community thoughts and little experience with user involvement.)
%		\end{itemize}
%	\item Conclusion
%		\begin{itemize}
%			\item The conclusion should highlight the current development process at TC Electronic and answer a research question.
%		\end{itemize}
%\end{itemize}

\noindent
Through the six phases of the thematic analysis \autoref{InterviewAnalysis} 272 codes were attached to data extracts from the interview conducted at TC \autoref{InterviewGuide}. These codes have then been grouped together in themes highlighting aspects of the interview that is found interesting or important. In total 33 themes were derived from the thematic analysis. The Description of the individual themes is seen in \autoref{App:Themes}, while the groupings of codes and data extracts which each theme is based on is located in the digital appendix \autoref{DigitalAppendix}\fxnote{Lav digitalt appendix med temaerne bygget op af codes, og lav en reference til det i digital appendix overviewet}. In order to use the themes to accommodate the purpose of the interview they are grouped as sub-themes under a overall label, which makes it more comprehensible. The sub-themes and overall labels is shown at \autoref{fig:ThemeLables}\fxnote{Lav figur}, in which the number in parenthesis indicates how many codes the theme is based on. 

\begin{figure}[H]
	\centering
	\includegraphics[width=0.4\textwidth]{Placeholder}
	\caption{This illustrates an overview of grouped Themes as sub-themes under a overall label}
	\label{fig:ThemeLables}
\end{figure}

\section{Findings from thematic analysis}
\label{ThematicFindings}
%
As seen in \autoref{ThemeLables} has the 33 themes been grouped under seven labels according to the scope of the themes. As indicated by the labels some groups does compare more directly to the interview research question e.g the groups 'Decision making' or 'Focus on the user'\fxnote{Det er den der lige nu heder User involvement/The user is in focus/Users}. However is the other groups of themes still found to be providing some useful information in regards to evaluate development of the TonePrint app. Each group is given a short description below, except the 'Miscellaneous' which includes codes and data extracts that isn't found useful for this study.\\
\subsubsection*{Decision making}
The themes in this group is all focusing on different aspects of how TC have made decisions through the development process of the TonePrint app. It's emphasized that most of their decisions have been based on gut-feeling towards what may be the best solution for the user. The design of the parameter settings interface were e.g. based on an assumption of it being the typical way of doing it. A repeated pattern is that the decisions haven't been based user studies. 

\subsubsection*{The workflow at TC}
These themes describes different aspects of how the work flow affected the development of the TonePrint app. An important point is how TC prioritize the features to be implemented. The weigh out how important a feature is found to be against how time consuming and resource demanding it is to implement. The product owner is in the end the one deciding what to implement and in instances also how it should be implemented. Further is it emphasized how useful they find the SCRUM framework and when tasks seems to large to fit a three week sprint it's deconstructed into smaller task.  

\subsubsection*{Focus on the user}
The themes in this group focus on how the user sometimes are in the focus of development, even though one theme again states that there have been no direct user studies focusing on the TonePrint app. The idea of the TonePrint system does however stem from a study with bass players making presets for at amplifier. One way they accommodate the users is by including feature requests they get through their web page or facebook, whereas these requests is added to their product backlog. They have a general idea of wanting to make their products user-friendly and when asked how it's suggested to "ask them what they want". It's stated that they have three target user to accommodate, pedal only users, artist TonePrint users and TonePrint creators. Though user might be fluctuating between thous.

\subsubsection*{Early stages of the TonePrint concept}
Through these themes it's seen that the TonePrint concept have evolved over time. It all began by taking former development tool used to creating parameter setting of a sound, which they at one point took used in collaboration with professional artist in order to make artist TonePrints that could be beamed to a pedal. One challenge have been to make this setting of parameters understandable for people who aren't sound engineers.

\subsubsection*{Current version of the TonePrint app}


\subsubsection*{External collaboration ind inspiration}


\subsubsection*{The TOnePrint community}



\section{The TonePrint development process}
\label{TonePrintDevelopmentProcess}
%
Through the thematic analysis of the interview an understanding of TC's development process of the TonePrint app have been derived. It's found that they in general haven't conducted any user studies as a part of their development process which may be a factor to why their representation of their target users is somewhat limited. This comes to show when talking of how decisions have been based mostly on gut-feeling e.g when implementing the new design of the app of the app. One way that user are involved by TC acting on feedback received though their website, music tribe community or their facebook group TonePrint junkies. Through these channels they receive feature request which is added to their product backlog, on the same level as their own ideas to new features. Another way these channels have been beneficial is when bugs are reported. But a repeated pattern found is that the users haven't been taken directly into account when designing the app.\\
When deciding whether an feature from the product backlog should be implemented it's comparison of importance versus time and resources. The TonePrint community is an example of this, where it was found somewhat important, because there were may requests for it, however was seen as a very time consuming and resourceful implementation. This resulted in TC waiting for a time where the necessary time is at hand. When it's decided to implement a feature it's divided into tasks that fits a three week sprint. This means that tasks to large to complete in one sprint is deconstructed in order to fit the design sprint. This way of using the SCRUM framework is praised by all in the interview and it's commented that it helps focus on the task at hand so that things that aren't important at the current state of the design process doesn't get unwanted attention.\\
When framing the attention against the development of the TonePrint community it's stated that it's important to ask the users what they want, which might be a very simplistic view on user centered design. Several ideas of features were suggested which further will be elaborated in \autoref{CommunityConcept}. It's apparent that there are different viewpoints towards what should be implemented in the TonePrint community, which further confirms that there should be conducted some studies focused on the future extend of the TonePrint community. 




%\newcolumntype{R}{>{\raggedright\arraybackslash}X} %Left adjust and no stretch of text. Good for narrow cells
%\begin{table}[H]
%\label{ThemesOverview}
%\small
%\begin{tabularx}{\textwidth}{|R|R|R|R|R|}
%\hline
%Beaming App & Ikke inspireret af andre & Udviklingsværktøj & Parameterdesign & Rollefordeling / Hieraki \\ \hline
%Prioritering af features & Beslutning på baggrund af test & Erfaring fra tidligere produkter & Målrettet mod brugergrupper & Beslutning på baggrund af antagelser \\ \hline
%Beslutning på baggrund af bekvemmelighed & Kommunikation i udvikling af app & Forretningsmodel & marketing \/ TonePrint beskrivelse & TonePrint koncept \\ \hline
% Parametre UI & Ingen brugerinddragelse & Beslutning på baggrund af personlige holdninger & Inspiration af eksterne Produkter & Rod \\ \hline
%UI design & Brugerinddragelse & Eksterne udviklere & Brugerfeedback & Brugerinddragelse \\ \hline
%Community efterspørgelse & Community tags & Tilbageholdt deling & Community beslutning & Feedback om langsom app \\ \hline
%Fokus på brugere og brugervenlighed & Prioritering af TonePrint appen & Samarbejde med kunstnere & Tekniske begrænsninger & Ingen målsætninger i forhold til TonePrint appen \\ \hline
%\end{tabularx}
%\caption{Themes overview}
%\end{table}

%%%%%
%Alt herfra indtil conclusion skal i appendix
%%%%%

%
%\section{Interview conclusions}
%\label{InterviewConclusion}
%Som resultat af analysen kan det ses at SCRUM er en meget vigitg del af mentaliteten hos TC og at den måde at opsætte krav for sprints og prioritere features, har ens stor betydning for deres udviklings process. Det virker til at den erfaring de har fra tidligere brugerinddragelse har være meget god, dog ikke den ene gang med Jesper, hvor det virker til at timingen har været forkert. Det virker til at de har en idé om hvem deres brugere er, dog uden helt at vide det, samtidig med at de ikke rigtig har erfaring med at målrette efter bestemte brugere når de designer, da de mener at TonePrint konseptet aspirere nok til deres brugere. Det ses at de har en masse idéer til TonePrint communitiet, hvor de fleste er enige om at det med Tags, er en vigtig del at få undersøgt og lavet.
