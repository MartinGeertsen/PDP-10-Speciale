\chapter{Results of thematic analysis}
\label{ThemanticAnalysis}

%\begin{itemize}
%\item Themes
%		\begin{itemize}
%			\item Present a overview of the themes
%			\item Link to appendix with a description of each theme
%			\item Describe the overall finding from the interview, with references to the Themes, which should refer to timestamps in the transcription.
%			\item (It’s important we highlight something related to SCRUM, decisions making, Community thoughts and little experience with user involvement.)
%		\end{itemize}
%	\item Conclusion
%		\begin{itemize}
%			\item The conclusion should highlight the current development process at TC Electronic and answer a research question.
%		\end{itemize}
%\end{itemize}

\noindent
Through the six phases of the thematic analysis (see \autoref{InterviewAnalysis}), 354 codes were created and attached to data extracts from the interview conducted at TC (see \autoref{InterviewGuide}). These codes have then been grouped together in themes highlighting different aspects of the interview. In total 33 themes were derived from the thematic analysis. A description of the individual themes is located in \autoref{App:Themes}, while the groupings of codes and data extracts, which each theme is based on, is located in \autoref{DigitalAppendix}. In order to easier get an overview of the themes, they are grouped as sub-themes under an overall label, which makes it more comprehensible. The sub-themes and overall labels are displayed on \autoref{fig:ThemeLables}.

\begin{figure}[H]
	\centering
	\includegraphics[width=\textwidth]{HorizontGroups.pdf}
	\caption{An overview of grouped Themes as sub-themes under a overall label}
	\label{fig:ThemeLables}
\end{figure}

\section{Findings from thematic analysis}
\label{ThematicFindings}
%
As displayed on \autoref{fig:ThemeLables}, the 33 themes have been grouped under eight labels according to their content. As indicated by the labels, some groups compare more directly to the interview research questions e.g the groups 'Decision making' or 'Focus on the user'. However, the other groups of themes are still found to be providing some useful information. Each group is given a short description below, except the 'Miscellaneous' which include codes and data extracts not fitting anywhere else.

\subsection*{Decision making}
The themes in this group are all focusing on different aspects of how TC have made decisions through the design process of the TonePrint app. It's emphasized that most of their decisions have been based on gut-feeling towards what may be the best solution for the users. The design of the parameter settings interface were e.g. based on an assumption of it being the typical way of doing it. A repeated pattern is that the decisions haven't been made on the basis of user studies. 

\subsection*{The workflow at TC}
These themes describe different aspects of how the work flow affected the development of the TonePrint app. An important point is how TC prioritize the features to be implemented. They weigh out how important a feature is found to be against how time-consuming and resource demanding it is to implement. The product owner is in the end the one deciding what to implement and in some instances, he also decides on how it should be implemented. Furthermore, it is emphasized how useful they find the SCRUM framework, and when tasks seems to large to fit a three week sprint it's deconstructed into smaller tasks.  

\subsection*{Focus on the user}
The themes in this group focus on how the user sometimes are in the focus of development, even though one theme again states that there have been no direct user studies focusing on the TonePrint app. The idea of the TonePrint system, however, stems from a study with bass players making presets for an amplifier. One way they aid the users is by including feature requests they get through their web page or facebook, where these requests are added to their product backlog. They have a general idea of wanting to make their products user-friendly, and when asked how, it's suggested to "ask them what they want". It's stated that they have three target users to aid, pedal only users, artist TonePrint users and TonePrint creators. Though user might be fluctuating between thous.

\subsection*{Early stages of the TonePrint concept}
Through these themes it's found that the TonePrint concept has evolved over time. It all began by utilising a former development tool, originally used for creating artist TonePrints in collaboration with the artists themselves. One challenge have been to make this development tool understandable for people who aren't sound engineers.

\subsection*{Current version of the TonePrint app}
The themes in this group sheds light on some concepts and issues of the current TonePrint app. One concept is the descriptions of the TonePrints, which currently are written by the marketing team. The descriptions are found to be an important aspect to include in the TonePrint community in order to ease the searching through what they expect to be a large amount of TonePrints. It's stated that there was a problem when the newest version of the TonePrint app was launched. The way the version was implemented across platforms resulted in problems for android users because the app was very slow in responding. Furthermore, it's also apparent that when redesigning the editor interface there was a focus of making it more user friendly, but there are no indications to how this user friendliness was defined or measured.

\subsection*{External collaboration ind inspiration}
This group focusses on TC's take on inspiration from external products and their use of external collaboration. It's apparent that there hasn't been any key source of inspiration for the development of the TonePrint app, and that it's developed as an original idea. It has been considered to look at similar products in the industry, though, when designing the TonePrint community in order to see what works and what doesn't. Finally, it is mentioned that the first TonePrint app was designed by an external designer, but the design was implemented by TC's own programmer.  


\subsection*{The TonePrint community}
This group consists of themes that describes different aspects of the TonePrint community regarding both features and overall considerations. The feature that's allocated most focus to is descriptive 'tags' for the TonePrints, which serve as small categorisation stamps. Other features such as 'user subscriptions' or a filtering system more advanced than in the current app are also seen as important to implement. The decision to develop the TonePrint community was made from many user requests of being able to share one's TonePrints withe other users. However, it isn't decided yet how it should be implemented. A point of which, there still is doubt, is whether or not it should be implemented as an extension of the current app or if it should be it's own thing.


\section{The TonePrint design process}
\label{TonePrintDevelopmentProcess}
Through the thematic analysis of the interview, an understanding of TC's design process of the TonePrint app have been derived. It's found that they in general haven't conducted any user studies as a part of their design process, which may be a factor to why their representation of their target users is somewhat limited. This comes to show when talking of how decisions have been based mostly on gut-feeling e.g when implementing the new design of the app. One way that users are involved is when TC act on the feedback they receive though their website, music tribe community and their facebook group \textit{TonePrint junkies}. Through these channels they receive feature request which is added to their product backlog, on the same level as their own ideas to new features. Another way these channels have been beneficial is when bugs are reported. A repeated pattern found is however still that the users haven't been involved directly during the design of the app. Deciding on whether a feature from the product backlog should be implemented is done from a mix of importance, and time and resources. The TonePrint community is an example of this, where it was found somewhat important, because there were may requests for it but was seen as a very time-consuming and resource demanding implementation. This resulted in TC deciding to wait until the time is available. When it's decided to implement a feature it's divided into tasks that fits a three week sprint. This means that tasks too extensive to complete in one sprint is deconstructed in order to fit the design sprint. This way of using the SCRUM framework is praised by all in the interview, and it's commented that it helps focus on the task at hand so that things that aren't important at the current state of the design process doesn't get unnecessary attention. When framing the attention against the development of the TonePrint community, it's stated that it's important to ask the users what they want, which might be a very simplistic view on human-centered design. Several ideas of features were suggested which further will be elaborated on in the following chapter. It's apparent that there are different viewpoints towards what should be implemented in the TonePrint community, which further confirms that there should be conducted some studies focused on the future of the TonePrint community. 




%\newcolumntype{R}{>{\raggedright\arraybackslash}X} %Left adjust and no stretch of text. Good for narrow cells
%\begin{table}[H]
%\label{ThemesOverview}
%\small
%\begin{tabularx}{\textwidth}{|R|R|R|R|R|}
%\hline
%Beaming App & Ikke inspireret af andre & Udviklingsværktøj & Parameterdesign & Rollefordeling / Hieraki \\ \hline
%Prioritering af features & Beslutning på baggrund af test & Erfaring fra tidligere produkter & Målrettet mod brugergrupper & Beslutning på baggrund af antagelser \\ \hline
%Beslutning på baggrund af bekvemmelighed & Kommunikation i udvikling af app & Forretningsmodel & marketing \/ TonePrint beskrivelse & TonePrint koncept \\ \hline
% Parametre UI & Ingen brugerinddragelse & Beslutning på baggrund af personlige holdninger & Inspiration af eksterne Produkter & Rod \\ \hline
%UI design & Brugerinddragelse & Eksterne udviklere & Brugerfeedback & Brugerinddragelse \\ \hline
%Community efterspørgelse & Community tags & Tilbageholdt deling & Community beslutning & Feedback om langsom app \\ \hline
%Fokus på brugere og brugervenlighed & Prioritering af TonePrint appen & Samarbejde med kunstnere & Tekniske begrænsninger & Ingen målsætninger i forhold til TonePrint appen \\ \hline
%\end{tabularx}
%\caption{Themes overview}
%\end{table}

%%%%%
%Alt herfra indtil conclusion skal i appendix
%%%%%

%
%\section{Interview conclusions}
%\label{InterviewConclusion}
%Som resultat af analysen kan det ses at SCRUM er en meget vigitg del af mentaliteten hos TC og at den måde at opsætte krav for sprints og prioritere features, har ens stor betydning for deres udviklings process. Det virker til at den erfaring de har fra tidligere brugerinddragelse har være meget god, dog ikke den ene gang med Jesper, hvor det virker til at timingen har været forkert. Det virker til at de har en idé om hvem deres brugere er, dog uden helt at vide det, samtidig med at de ikke rigtig har erfaring med at målrette efter bestemte brugere når de designer, da de mener at TonePrint konseptet aspirere nok til deres brugere. Det ses at de har en masse idéer til TonePrint communitiet, hvor de fleste er enige om at det med Tags, er en vigtig del at få undersøgt og lavet.
