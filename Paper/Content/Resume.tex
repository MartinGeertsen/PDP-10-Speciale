\begin{LARGE}
\textbf{Resume}
\end{LARGE}
\\
\noindent
Dette projekt er skrevet i samarbejde med virksomheder TC Electronic (TC) som står bag udviklingen af TonePrint konceptet og den dertilhørende TonePrint app. TonePrint appen tillader dens brugere at overføre guitar eller bas effekter (TonePrints) til ens TonePrint pedal. Man kan enten vælge at overføre prædefineret TonePrints som der er lavet i samarbejde mellem TC og en professionel musiker, eller man kan lave TonePrintne selv, ved hjælp af appens editerings funktion. Projektet tager udgangspunkt i at TC skal til at lave en platform hvor brugere kan dele de TonePrints de selv har lavet (TonePrint community) og til denne udvikling vil de gerne gøre brug af brugercentrerede udviklings metoder. Ud fra en introduktion udledes der en problemformulering som oversat sig ”Hvordan kan metoder for en brugercentreret designtilgang bruges i designprocessen af TonePrint communityet.” \\
\\
Det besluttes at undersøge designprocessen for TonePrint appen, fordi det er det produkt som der har flest ligheder med det kommende TonePrint community. For at lave undersøgelse forberedes et semistruktureret interview som bliver afviklet med fire ansatte hos TC som har været en del af udviklingsteamet for TonePrint appen. Som forberedelse til interviewet udføres en heuristisk evaluering af appen, som har til formål at give projektets forfattere en bedre indsigt i TonePrint appen. \\
Efter interviewende er blevet udført, laves der er tematisk analyse af resultaterne, hvorigennem der fremstilles 33 temaer baseret på 354 koder fra det transskriberet interview data. Temaerne fremhæver hver især nogle pointer fra interviewet. De samles i nogle grupper og bruges til at beskrive designprocessen af TonePrint appen. Der findes frem til at SCRUM-arbejdsprocessen er et vigtigt element for deres designproces og at der ikke tidligere har været lavet brugerstudier som en del af designprocessen. Ud fra interviewet bliver der også lavet et overblik over idéerne for det kommende TonePrint community, hvor forskellige funktioner og koncepter beskrives og der opstilles en konceptuel model for TonePrint communityet, samtidig med at der defineres to brugs scenarier. \\
Den akkumulerede viden anvendes til at opstille fem opgaver der fokusere på forskellige område af udviklingen af communityet. Disse opgaver bruges til at konkretisere hvordan brugercentret designmetoder kan anvendes med henblik på udviklingen af TonePrint communityet. I beskrivelsen af disse opgaver fremhæves de specifikke metoder der vurderes relevante og der reflekteres over hvordan disse kan tilpasses til en design sprint, som er en del af SCRUM-arbejdsprocessen. \\
\\
Det vælges at udføre en af opgaver, som fokusere på at undersøge brugernes mentale model for den senere kan anvendes til designprocessen af informationsstrukturen af TonePrint communityet. For at imødekomme denne opgave udføres der et card sort forsøg og en \textit{participatory design} workshop som til sammen skal belyse brugernes mentale model a TonePrint communityet.\\
Resultaterne fra card sorten bliver meget begrænset da der kun er seks besvarelser uden noget reelt kvalitativt data, da den er udført som et remote study. Workshoppen bliver udført med 5 brugere af TonePrint appen, hvori de først individuelt skal prøve at skitsere deres mental model for udførelsen af en opgave med TonePrint communityet, for bagefter at lave blive enige om en model for løsningen på en anden opgave. \\
De to forsøg resultere i en model som afbilleder brugernes mentale model af TonePrint communityet, i det omfang der bruges til at udføre de to opgave fra workshoppen, som svare til de to brugsscenarier defineret tidligere ud fra interviewet.
