\chapter{Exploring the TonePrint application}
\label{HeursiticEvaluation}
Before being able to help TC Electronic with using user centered design methods in their development process, it's necessary to examine the current trend of the development process. Different methods could be used to examines this process, which all have advantages and disadvantages. Some of the methods will be discussed, before selecting one.

\subsection*{Observational study}
\label{MethodObservation}
An observational study is a classic method used in field studies. This could be used to observe the entire development team, in order to analyze their every day work process, and the overall development process. This allows the observer to create a understanding of development process, which isn't biased by the individuals of the development team, whom might have problems recalling every aspects of the process. However would a classic observational study not allow the observers to gain knowledge of the background of the individual team members decisions. There is however a method which have many things in common with classical observational studies, which is contextual inquiry. In contextual inquiry your normally observing the use of a product in the target environment and by it's target user, to analyze usability. Besides the observation part of this method, it also relay on the observer to interview the user, while he or she is interacting with the product. This enables the observer to get a better understanding of what user are thinking. This could be used in observing the development process, where the process wold be observed, and team members would be interviewed when the observers would find it relevant. The disadvantage with this method is however that the development process is quit long. An observational study will take to time consuming, if it should be able to cover the broadness of the development process.

\subsection*{Workshop}
\label{MethodWorkshop}
To accommodate the problem of having to make the investigation 

Her skal det beskrives hvordan en workshop kunne bruges til at analysere TC's udviklings proces. Gennem en workshop wille man kunne få de forskellige udviklere til at udføre opgaver meget lig, opgaver der kunne fremkomme i udviklings processen, for at se hvordan de griper dem and. De kunne også blive sat til at "Spille" sig selv, for at derved at vise hvordan de gjort. En ulempe ved denne metode er dog at det ville være at dataen måske ikke er så pålidelige, som ved eventuelle interview hvor de selv siger hvordan de har gjort.

\subsection*{Interview}
\label{MethodInterview}
Her skal det beskrives at et interview med TC's udviklere ville kunne give et billede på hvordan de har udviklet tidligere produkter, for at få et indblik i deres proces. Her vil det være logisk at tage udgangspunkt i TonePrint appen, da den har flere aspekter til fælles med community idéen. Ved interviews kan det være en ulempe at nogle ting kan være glemt og at nogle spørgsmål passer bedre til forskellige udviklere, da de har siddet med forskellige opgaver, hvilke vi vil imødekomme med at lave det semistruktureret, så vi kan gå nemt i dybten med hvad de enkelte har haft fokus på. 

\begin{itemize}
	\item Vi vil gerne have en bedre forståelse af TonePrint appen, hvorfor det?
	\begin{itemize}
		\item Generelt skal det bruges til at forberede os på interviewet.
		\item Vi leder efter faldgrupper i appen, som vi kan snakke om i interviewet.
		\item Vi ved altså allerede på det her tidspunkt, at vi har tænkt os at lave interviews.
	\end{itemize}
	\item Vi leder efter forskellige metoder til dette formål
	\begin{itemize}
		\item Er det usability, UX eller noget tredje, vi leder efter?
		\item For usability/UX kan man lave brugerinddragelse
		\item Dette er dog tidskrævende
		\item I stedet kan man lave en heuristisk evaluering
		\item Alternative metoder til heuristisk evaluering?
	\end{itemize}
	\item Vi går med at lave en heuristisk evaluering
	\begin{itemize}
		\item Formålet - Vi skal udpege faldgrupperne
		\item Udover, at det hjælper os, fungerer det også som et studie af appen for dem.
		\item Derfor skal vi overveje, hvordan vi beskriver problemerne, så det er gavnligt.
		\item Hvad er formålet med de forskellige platforme?
	\end{itemize}
	\item Resultat og analyse
	\begin{itemize}
		\item De cirkulære slidere - Bruger vi den rigtige analogi?
		\item Rop synes, vi også bør udpege de ting, der fungerer godt.
	\end{itemize}
	\item Konklusion - Hvordan hjælper dette os i forhold til interviewet?
	\begin{itemize}
		\item Vandt nogle forskelle på tværs af platforme.
		\item Tyder på en løs tilgang til det og måske en mangel på kommunikation internt
	\end{itemize}
\end{itemize}

\section{Heuristic evaluation}
\label{SectionHeuristicEvaluation}
For at kunne stille de bedst mulige spørgsmål om udviklingen a TonePrint appen vælger vi at lave en heursitks evaluering af appen. Den vil hjælpe med at kaste lys over usability problemer, som vi så kan spørge ind til. Her vil det muligvis blive tydeligt hvordan en manglende brugerinddragelse har haft en effekt på deres udvikling. \\
I denne sektion vil teorien, fremgangsmåden og heuristikkerne blive forklaret.

\section{Heuristic Evaluation Results}
\label{Heuristic_Results}
The results of the heuristic evaluation are presented in categories of what usability heuristic they may violate....

\subsubsection{Visibility of system status}
\begin{itemize}
	\item When browsing through the available TonePrints for artists, some of them may have created the same TonePrint settings for multiple pedals. Clicking between these doesn’t provide any clear feedback to which is selected however, as the description of the TonePrint is the same whichever pedal it is set for.\\
	\item There is a lack of indication to which instrument is selected, as this selection happens in settings and not in the list itself. If either \textit{guitar} or \textit{bass} is selected under the instrument filter, and not \textit{all}, the message in the list \textit{"all TonePrints by…”} is misleading, as the user is only going to find TonePrints for one instruments.\\
	\item When pressing \textit{user} on the computer application there are no indications of what to do next. The user is just presented a blank column with nothing in it.\\
	\item When selecting the \textbf{Helix Phaser} with the \textit{guitar} filter active on the computer app, nothing happens. When trying this on the Iphone app, it opens one TonePrint, and when opening it on an android unit, the app crashes.\\
	\item When pressing the video icon on the android and computer app, it isn't clear that the unit will open youtube in a web browser compared to the Iphone app.
\end{itemize}
%
\subsubsection{Match between systems and the real world}
\begin{itemize}
	\item The sliders for the various parameters are all presented as circular sliders, but interaction with them are done by pressing the center of it and swiping up or down. As such there is a risk of grabbing the entire canvas and not the parameter in question.\\
	\item It appears to still be possible to select bass TonePrints with the \textit{guitar} filter active.
\end{itemize}
%
\subsubsection{User control and freedom}
\begin{itemize}
	\item Nothing here...
\end{itemize}
%
\subsubsection{Consistency and standards}
\begin{itemize}
	\item Some artists have published the same TonePrint for multiple pedals and when switching between these, the text description is the same. However, in some cases there is a noticeable difference when doing these switch, as some of the descriptions has minor spelling or typeset errors, even though they should be identical.\\
	\item When opening a video description of a TonePrint with its creator on the smartphone app, it is presented in a new window. When opening one in the computer app, it passes you on to the given video on youtube. \\
	\item When browsing TonePrints, there are different buttons in the top right corner of the description page, depending on on the TonePrint.\\
	\item When watching a video description of a TonePrint on the Iphone app and the user at some point wants to return to the list of TonePrints or artists, it demands two different interactions. First, the user must swipe down in order to return to the TonePrint description, before either swiping right or pressing \textit{back} to get back to the list view.\\
	\item When choosing the \textbf{SpectraComp Bass Compressor} with the \textit{guitar} filter on, the user dosen't get the same menu as when choosing other pedals. This is probably due to it being a bass effect.\\
	\item When creating a favourites list, the TonePrints are sorted by pedal name, even if the user selects \textit{sort by artist}.\\
	\item When opening the app on an android unit, the user gets informed that he needs a midi connection. This message doesn't appear on the desktop version, even though the same goes for that.\\
	\item The user has a search functionality available on the android system but not on either the desktop or Iphone version.
\end{itemize}
%
\subsubsection{Error prevention}
\begin{itemize}
	\item The typical confirmation dialogue of either \textcolor{xGreen}{\textbf{$\checkmark$}} or \textcolor{xRed}{\textbf{$\times$}} is presented to the users with these icons inside the button on the Iphone app. As such it isn’t clear whether the user selects an action when it is visible, or if this visibility means that it is already selected.\\
	\item When the user is beaming a TonePrint to the pedal, he is given the instruction: \textit{If your pedal flashed like this beaming was a succes}. In order to follow this instruction the user would have to focus on the pedal, and by doing this he wouldn't have seen this instruction in the first place. As such, the user has to focus on two things at once.\\
	\item The user can assign different parameters to the same physical button on the pedal, allowing for live editing of the TonePrint. However, the pedal comes with a print above the knob on the pedal itself, which can't change. As such, the user can potentially edit a parameter, even though the knob says something different.
\end{itemize}
%
\subsubsection{Recognition rather than recall}
\begin{itemize}
	\item When switching between \textit{browse by product} and \textit{browse by artist}, this has to be done under settings, and the same goes for switching between type of instrument. Instead of having this filtering action visible with the list, the user must remember to check this in the settings menu. 
\end{itemize}
%
\subsubsection{Flexibility and efficiency of use}
\begin{itemize}
	\item In general there are limited ways of customising the canvas, for example the favourite list.\\
	\item The search functionality on the android app only allows for searching in the open menu, making it almost redundant. The user still needs to to go to the right menu before searching for specifics, making scrolling a faster way of finding the right TonePrint.
\end{itemize}
%
\subsubsection{Aesthetic and minimalist design}
\begin{itemize}
	\item It’s limited to what extend the size of the canvas can be expanded on the computer app. If it is made full-screen it will no long match the size of the window and take all the space. Instead, the far right of the window will just be a blank column of nothing.\\
	\item When opening the computer application, until something is chosen, the screen will primarily be just blank.
\end{itemize}
%
\subsubsection{Help users recognise, diagnose, and recover from errors}
\begin{itemize}
	\item Nothing here...
\end{itemize}
%
\subsubsection{Help and documentation}
\begin{itemize}
	\item When choosing \textit{Editor Help}, the user is sent to the main TonePrint webpage.
\end{itemize}
%







