\chapter{Evaluating the design process of the TonePrint system}
\label{HeuristicEvaluation}
As explained in \autoref{Introduction} and established in the research question (see \autoref{ResearchQuestion}), the purpose of this project is to aid TC Electronic towards applying a human-centered design approach to their design process of a future TonePrint community. This will be done by first investigating the design process of an existing product in the realm of the TonePrint community i.e. their current TonePrint app. This choice is made on the basis of the expected difference in the design processes, depending on whether the product in focus is hardware or software based, and furthermore, the TonePrint app lies close to the TonePrint community. The following chapter focusses on this task, starting with choosing an approach for getting a better understanding of the TonePrint app, before moving on to getting a better understanding of the design process of it.

\section{Approach}
\label{AnalyzingTonePrintDesign}
For the analysis of the design process of the TonePrint application, different approaches can be taken, and these should therefore be taken into account. An important speculation is, whether the analysis should be based on observable or self-reported data, and these two approaches will therefore be examined.


\subsection*{Observational study}
\label{MethodObservation}
An observational study is a classic method used in field studies. This can be used to observe the entire development team, in order to analyse their every day work process and the overall design process. This allows the observer to create an understanding of the design process, which isn't biased by the individuals of the development team, who might have problems recalling every aspects of the process. However, a classical observational study won't allow the observers to gain knowledge of the background of the individual team members decisions. For this, there instead is a method with many things in common with classical observational studies. This is called a \textit{contextual inquiry}. In contextual inquiries you're normally observing the use of a product by it's target user in the right setting in order to analyse usability. Besides the observation part of this method, it also relies on the observer to interview the user, while he or she is interacting with the product. This enables the observer to get a better understanding of what the user is thinking. This method could be utilised in this project by engaging in observing the design process and then interview the team members, when the observers would find it relevant. One problem with this method, however, is that it might be too time-consuming for this project. As mentioned in \autoref{scrum}, TC utilises the SCRUM framework, and getting a better understanding of their design process would therefore require observing one of these sprint periods, and preferably more. Another problem is that a design process consists of several phases of meetings, iterations, and rethinking design ideas, which is difficult to cover adequately.


\subsection*{Self-reporting methods}
\label{MethodInterview}
Compared to the observational methods, there is a large branch of methods which rely on getting the involved personals' own reflective view on the process. When looking at the aspects which is found problematic for evaluating the design process through observations, these wouldn't necessarily be problems for a self-reporting inquiry. The long span of the design process, and the iterative process which the TonePrint app already have gone through, could be explored by getting the people, who have been a part of the process, to elaborate on it. This of course relies heavily on the memory of the respondents and their willingness and ability to reflect upon former decisions and processes. There are many different approaches which may be used to examine the design process using the involved personals reflective knowledge. In one end of the spectrum there is the highly structured survey presenting close-ended questions, whereas in the other end there is the unstructured focus-group inquiries, focusing on discussing open-ended questions in smaller groups \parencite{WEB:ConductingSemiStructured}. A middle ground between these approaches would arguably be the semi-structured interview approach. The scope of this approach is to use both open and close-ended questions for further clarifications than close-ended would provide on their own. By using the approach of a semi-structured interview to analyse the design process of the TonePrint app, it allows the individual interviews to better focus on areas found interesting, based on the answers provided by the subjects. The interview will more or less become a conversation in which the subjects describe their view on the design process based on questions prepared before the interview, and questions created on the go to further clarify the answers to the prepared questions. Taking this into account, it's decided to conduct this analysis using the approach of the semi-structured interview.

\subsection{Preparing the interview}
\label{PreparingTheInterview}
When preparing an interview, it's important to establish what purpose the interview serves, which may by outlined in the form of research questions. These research questions serve as objectives for the interview, anf they are as followed:
\begin{enumerate}
	\item What is the TonPrint apps developers understanding of their target users and how was it applied in the design of the App?
	\item How was the decision process when deciding on, and designing new implementations for the TonePrint App?
	\item How did the SCRUM framework affect the design process of the TonePrint app?
	\item Which aspects does the development team find important for the design process of the TonePrint community?
\end{enumerate}

\noindent
In order to meeting the purpose of the interview, the research questions above are used as guidelines for generating the final questions for the interview. Before these questions are generated, it's necessary to establish a ground knowledge of the TonePrint app and how it's designed. This will help framing the questions towards being more context specific which might help avoiding too arbitrary and non specific answers, that otherwise would be difficult to interpret. By properly framing these questions, they may also help frame appropriate follow-up questions on the go if deemed necessary. In order to obtain this knowledge, an exploration of the TonePrint app should be conducted.

\section{Heuristic evaluation}
\label{SectionHeuristicEvaluation}
As mentioned in \autoref{PreparingTheInterview}, it's necessary to explore the TonePrint app before conducting the interview. When exploring the TonePrint app it's also decided to create a usability evaluation. The objective of the evaluation is to highlight usability problems that may be addressed in the interview. This enables questions to be directed at specific parts of the app, hopefully resulting in answers that describe the decisions related to the development of specific parts of the app.

Many different methods and approaches exists for evaluating the usability of a system, which differ in thoroughness, time and resources. Given that the scope of this evaluation is to generate questions and explore the application, it's decided to use a method that is fast to plan, conduct and analyse. A method, which is deemed applicable with the given terms, is the Heuristic evaluation, which is described by \parencite{WEB:Nielsen1994HowTo}. The scope of this method is to have a team of evaluators, preferably with some expertise regarding usability design, evaluate the usability of a systems interface, by comparing elements with a set of usability design heuristics. The result of the heuristic evaluation is a list of usability problems, defined by the heuristic they violate, and why. \textcite{WEB:Nielsen1994HowTo} recommends that the number of evaluators should be among three or five. It's stated that less than three evaluators likely won't be able to identify a sufficient number of usability problems, while more than five evaluators will have problems identifying new usability problems, that haven't already been identified by another evaluator. The use of the evaluation in this study is however not to identify all usability problems in order to redesign the interface but to find examples to base the interview questions on. The heuristic evaluation is therefore still considered applicable, even though the only evaluators are the two authors of this thesis.
\newpage
\noindent
A key factor of the heuristic evaluation is obviously the heuristics which are used for identifying the usability problems. \textcite{WEB:Nielsen1994} defines nine usability heuristics which identifies different kinds of usability problems. The heuristics were created by analysing seven sets of existing usability heuristics, with a combined total of 101 heuristics. The paper resulted in nine usability heuristics that can be used in a heuristic evaluation, however, while referring to his own heuristics \parencite{WEB:Nielsen1994} Jakob Nielsen describes ten heuristics \parencite{WEB:Nielsen1994Ten,WEB:Nielsen1994HowTo}, adding One more to the list. These ten heuristics are used for this evaluation and is listed in \autoref{Tab:HeuristicsOverview}.

\begin{table}
\centering
\begin{tabular}[width=\textwidth]{cl}
\hline
& \textbf{Heuristics \parencite{WEB:Nielsen1994Ten} }\\ \hline
1 & Visibility of the system status \\ 
2 & Match between system and the real world \\ 
3 & User control and freedom \\ 
4 & Consistency and standards \\ 
5 & Error Prevention \\ 
6 & Recognition rather than recall \\ 
7 & Flexibility and efficiency of use \\ 
8 & Aesthetic and minimalist design \\ 
9 & Help users recognize, diagnose, and recover from errors \\ 
10 & Help and documentation \\ \hline
\end{tabular}
\caption{The Ten usability heuristic described by \textcite{WEB:Nielsen1994Ten}}
\label{Tab:HeuristicsOverview}
\end{table}

\subsection{Procedure}
\label{HeuristicProcedure}
By following the steps described by \textcite{WEB:Nielsen1994HowTo}, the evaluation is divided into two phases. The scope of the first phase is for the evaluators to get familiar with the system they are evaluating. Here they are to inspect and navigate through the system and learn the way around it. The purpose of this is to prevent identifying false problems due to a lack of knowledge of the system and to ensure that the evaluators don't miss any areas of the system, while identifying usability problems. The second phase is the evaluation phase, where the evaluators find usability problems by comparing different elements of the interface to the usability heuristics \autoref{Tab:HeuristicsOverview}. When something in the interface violates a heuristic, it's noted by the evaluator which in the end leads to a list of violations of the heuristics, hence usability problems. 

\subsection{Evaluation Results}
\label{Heuristic_Results}
Based on the procedure described above, each evaluator noted the identified usability problems by shortly describing it and noting the heuristic it violates. The problems found by each evaluator is compared and combined, and these results are presented in \autoref{AppendixHeuristics}. 

\subsection{Discussion}
\label{HeuristicDiscussion}
%
When looking at the findings of the heuristic evaluation in \autoref{AppendixHeuristics}, it's important to keep in mind that the purpose of this project isn't to accommodate the usability problems in a redesign of the TonePrint app. The purpose of this evaluation is to explore the TonePrint app in order to obtain the necessary insight for making the interview context specific. One thing that might have affected the findings of this evaluation is that the two evaluators aren't necessarily end-users of the system. This might have resulted in problems identified from a lack of knowledge regarding domain specific design patterns. This further highlights the interesting point of how TC have directed their design towards their target users, or how the users otherwise have been considered during the design process, which is mentioned in \autoref{PreparingTheInterview}. A variety of problems were concerning overall problems with the evaluators not finding the information expected through a series of interactions. This should lead to questions inquiring the decisions regarding the structure and design of information though the app. Some more specific problem that should be addressed is the problems which seems to be platform specific, either for PC, MAC, Android or Iphone. This is interesting because it might enlighten how design decisions is made or how different users are targeted though the design.






