\chapter{Exploring the TonePrint application}
\label{HeursiticEvaluation}
\begin{LARGE}
Evaluating the TonePrint development
\end{LARGE}

\begin{itemize}
	\item What’s the purpose?
		\begin{itemize}
		\item We want to know how they formerly have developed software, so that we have a better understanding of were to make changes.
		\item We chose the TonePrint app because it’s a starting point for the Community and it’s an application from TC which draws parallels to the future community
		\end{itemize}
	\item How can we do it?
		\begin{itemize}
		\item Observation is an option, it would however take long time, and would only be able on products which currently is developed.
		\item Interview is a smart way because it’s easy to perform and adapt.
		\end{itemize}
	\item Our choice
		\begin{itemize}
		\item We want to make a semi structured interview with the development team.
		\item The interview is focusing on the development of the TonePrint App.
		\item We also want to know about thoughts concerning the TonePrint Community.
		\end{itemize}
\end{itemize}

\begin{LARGE}
Evaluating the TonePrint application
\end{LARGE}

\begin{itemize}
	\item What’s the purpose?
		\begin{itemize}
		\item We want to make a quick usability evaluation of the TonePrint App, to better aim the interview questions towards real problems.
		\item It can be fast, of cause on the cost of precision and amount of problems, however is it still found suitable.
		\end{itemize}
	\item How can we do it?
		\begin{itemize}
		\item Two evaluators
		\item Discussion of heuristics
		\end{itemize}
	\item Heuristic evaluation theory and approach
		\begin{itemize}
		\item Describe the theory of using experts and how it should be two sessions
		\end{itemize}
	\item Reslts
		\begin{itemize}
		\item The results are grouped accordingly to the heuristics.
		\end{itemize}
	\item Conclusion
		\begin{itemize}
		\item Discuss the findings and how questions can be drawn from them.
		\end{itemize}
\end{itemize}

Before being able to help TC Electronic with using user centered design methods in their development process, it's necessary to examine the current trend of the development process. Different methods could be used to examines this process, which all have advantages and disadvantages. Some of the methods will be discussed, before selecting one.

\subsection*{Observational study}
\label{MethodObservation}
An observational study is a classic method used in field studies. This could be used to observe the entire development team, in order to analyze their every day work process, and the overall development process. This allows the observer to create a understanding of development process, which isn't biased by the individuals of the development team, whom might have problems recalling every aspects of the process. However would a classic observational study not allow the observers to gain knowledge of the background of the individual team members decisions. There is however a method which have many things in common with classical observational studies, which is contextual inquiry. In contextual inquiry your normally observing the use of a product in the target environment and by it's target user, to analyze usability. Besides the observation part of this method, it also relay on the observer to interview the user, while he or she is interacting with the product. This enables the observer to get a better understanding of what user are thinking. This could be used in observing the development process, where the process wold be observed, and team members would be interviewed when the observers would find it relevant. The disadvantage with this method is however that the development process is quit long. An observational study will take to time consuming, if it should be able to cover the broadness of the development process.

\subsection*{Workshop}
\label{MethodWorkshop}
To accommodate the problem of having to make the investigation 

Her skal det beskrives hvordan en workshop kunne bruges til at analysere TC's udviklings proces. Gennem en workshop wille man kunne få de forskellige udviklere til at udføre opgaver meget lig, opgaver der kunne fremkomme i udviklings processen, for at se hvordan de griper dem and. De kunne også blive sat til at "Spille" sig selv, for at derved at vise hvordan de gjort. En ulempe ved denne metode er dog at det ville være at dataen måske ikke er så pålidelige, som ved eventuelle interview hvor de selv siger hvordan de har gjort.

\subsection*{Interview}
\label{MethodInterview}
Her skal det beskrives at et interview med TC's udviklere ville kunne give et billede på hvordan de har udviklet tidligere produkter, for at få et indblik i deres proces. Her vil det være logisk at tage udgangspunkt i TonePrint appen, da den har flere aspekter til fælles med community idéen. Ved interviews kan det være en ulempe at nogle ting kan være glemt og at nogle spørgsmål passer bedre til forskellige udviklere, da de har siddet med forskellige opgaver, hvilke vi vil imødekomme med at lave det semistruktureret, så vi kan gå nemt i dybten med hvad de enkelte har haft fokus på. 

%\begin{itemize}
%	\item Vi vil gerne have en bedre forståelse af %TonePrint appen, hvorfor det?
%	\begin{itemize}
%		\item Generelt skal det bruges til at forberede os på interviewet.
%		\item Vi leder efter faldgrupper i appen, som vi kan snakke om i interviewet.
%		\item Vi ved altså allerede på det her tidspunkt, at vi har tænkt os at lave interviews.
%	\end{itemize}
%	\item Vi leder efter forskellige metoder til dette formål
%	\begin{itemize}
%		\item Er det usability, UX eller noget tredje, vi leder efter?
%		\item For usability/UX kan man lave brugerinddragelse
%		\item Dette er dog tidskrævende
%		\item I stedet kan man lave en heuristisk evaluering
%		\item Alternative metoder til heuristisk evaluering?
%	\end{itemize}
%	\item Vi går med at lave en heuristisk evaluering
%	\begin{itemize}
%		\item Formålet - Vi skal udpege faldgrupperne
%		\item Udover, at det hjælper os, fungerer det også som et studie af appen for dem.
%		\item Derfor skal vi overveje, hvordan vi beskriver problemerne, så det er gavnligt.
%		\item Hvad er formålet med de forskellige platforme?
%	\end{itemize}
%	\item Resultat og analyse
%	\begin{itemize}
%		\item De cirkulære slidere - Bruger vi den rigtige analogi?
%		\item Rop synes, vi også bør udpege de ting, der fungerer godt.
%	\end{itemize}
%	\item Konklusion - Hvordan hjælper dette os i forhold til interviewet?
%	\begin{itemize}
%		\item Vandt nogle forskelle på tværs af platforme.
%		\item Tyder på en løs tilgang til det og måske en mangel på kommunikation internt
%	\end{itemize}
%\end{itemize}

\section{Heuristic evaluation}
\label{SectionHeuristicEvaluation}
As mentioned in \autoref{MethodInterview} it's necessary to explore the TonePrint app before conducting the interview. When exploring the TonePrint app it's also decided to create a usability evaluation. The objective of the evaluation is to highlight usability problems that may be addressed in the interview. This enables questions to be directed at specific parts of the app, hopefully results in answers that describes the decisions related to the development of specific parts of the app. \\
There exist many different methods and approaches for evaluating the usability of a system, which differ in thoroughness, time and resources\footnote{Skal vi bruge en kilde til dette?}. Given that the scope of this evaluation is to generate questions and explore the application, it's decided to use a methods that is fast to plan, conduct and analyze. A method which is deemed applicable with the given terms, is the Heuristic evaluation, which is described by \parencite{WEB:Nielsen1994HowTo}. The scope of this method is to have a team of evaluators, preferable with some expertise regarding usability design, evaluating the usability of a systems interface, by comparing elements with a set of usability design Heuristics. The result of the heuristic evaluation is a list of usability problems, defined by how the heuristic they violate, and why. \textcite{WEB:Nielsen1994HowTo} recommends that the number of evaluators should be among Three to Five. It's stated that less than Three evaluators probably wouldn't be able to identify a sufficient number of usability problems, while more than Five evaluators would have problems identifying new usability problems, that haven't allready been identified by another evaluator. The use of the evaluation in this study is however not to identify all usability problems in order to redesign the interface, but to find examples that some of the interview would be based on. The heuristic evaluation is therefore still considered applicable, even though the only evaluators is the Two authors of this thesis.\\ 
A key factor of the heuristic evaluation is obviously the heuristics which is used for identifying the usability problems. \textcite{WEB:Nielsen1994} defines nine usability heuristics which identifies different aspects kinds of usability problems. The heuristics was created by analyzing seven sets of existing usability heuristics, with a combined total of 101 heuristics. The paper results in Nine usability heuristics that may be used in a heuristic evaluation, however while referring to hes own heuristics \parencite{WEB:Nielsen1994} Jakob Nielsen describes Ten heuristics \parencite{WEB:Nielsen1994Ten,WEB:Nielsen1994HowTo}, adding One more to the list. These Ten heuristics is used for this evaluation and is listed in \autoref{Tab:HeuristicsOverview}\fxnote{Skal vi have en stører diskussion vedrørende huristikkerne?}.

\begin{table}
	\centering
\begin{tabular}[width=\textwidth]{cl}
\hline
& \textbf{Heuristics \parencite{WEB:Nielsen1994Ten} }\\ \hline
1 & Visibility of the system status \\ 
2 & Match between system and the real world \\ 
3 & User control and freedom \\ 
4 & Consistency and standards \\ 
5 & Error Prevention \\ 
6 & Recognition rather than recall \\ 
7 & Flexibility and efficiency of use \\ 
8 & Aesthetic and minimalist design \\ 
9 & Help users recognize, diagnose, and recover from errors \\ 
10 & Help and documentation \\ \hline
\end{tabular}
\caption{The Ten usability heuristic described by \textcite{WEB:Nielsen1994Ten}}
\label{Tab:HeuristicsOverview}
\end{table}

\subsection{Procedure}
\label{HeuristicProcedure}
By following the steps described by \textcite{WEB:Nielsen1994HowTo} the evaluation is divided into two phases. The scope of the first phase is for the evaluators to get familiar with the system they are evaluating. Here they are to inspect and navigate through the system and learn the way around it. The purpose of this is to prevent identifying false problems due to a lack of knowledge of the system and to ensurer that the evaluators doesn't miss any areas of the system, while identifying usability problems. The second phase is the Evaluation phase, where the evaluators finds usability problems by comparing different elements of the interface to the usability heuristics \autoref{Tab:HeuristicsOverview}. When something in the interface violates a heuristic, it's recorded by the evaluator which in the end leads to a list of violations of the heuristics, hence usability problems. 

%For at kunne stille de bedst mulige spørgsmål om udviklingen a TonePrint appen vælger vi at lave en heursitks evaluering af appen. Den vil hjælpe med at kaste lys over usability problemer, som vi så kan spørge ind til. Her vil det muligvis blive tydeligt hvordan en manglende brugerinddragelse har haft en effekt på deres udvikling. \\
%I denne sektion vil teorien, fremgangsmåden og heuristikkerne blive forklaret.

\subsection{Evaluation Results}
\label{Heuristic_Results}
Based on the procedure described above, each evaluator recorded they usability problems they identified by shortly describing the problem and noting the heuristic it violates. The problems found by each evaluator is compared and combined. The combined results are presented in \autoref{AppendixHeuristics}. 

\subsection{Discussion}
\label{HeuristicDiscussion}
When looking at the usability problems listed in \autoref{AppendixHeuristics} it's important to remember the scope of conducting this evaluation. The scope is not to describe all the systems usability issues in order to redesign of the TonePrint application, but rather to generate the necessary knowledge regarding the app to specify questions and discussions towards the process of developing of the app. \\
One thing that might have affected the findings of this evaluation is that the two evaluators isn't necessarily target users of the system. This might have resulted in problems identified based on a lac of knowledge regarding domain specific design patterns. This highlights a interesting point of view which should be explored though the interview, namely how they have directed their design towards their target users or how thy otherwise have been considered through the design process. \\







