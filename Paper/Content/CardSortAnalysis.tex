\chapter{Card sort analysis}
\label{CardSortAnalysis}
As it is explained in \autoref{AnalysisMethod}, there are two ways of approaching the analysis of a card sort, exploratory or statistical. It largely depends on the goal of the card sort and the amount of responses from subjects. For this card sort, the goal was to explore the general idea of how the users of the TonePrint system would group concepts of the TonePrint community. This includes what kind of groups they would create, and what concepts they found that belonged together. Ultimately, only six subjects completed the card sort, limiting the extend to which it can be analysed. Nonetheless, the following chapter will present these results of the card sort, including an analysis of it.

\section{Summary of Results}
\label{SummaryResults}
The six subjects completing the card sort are all male, aging from 26 to 44 years old with an average of 34.7 years, and they furthermore all come from different countries. Through the card sort the subject have constructed a combined total of 32 groups of varying sizes which is displayed in \autoref{Tab:GroupOverview}.

Before engaging with the analysis, it's desired to create an overview of how the individual cards that have been sorted. This is done in a spreadsheet containing all the cards in one column with the name of the groups in the adjacent columns (see \autoref{DigitalAppendix}). The next step, as described by \textcite[184]{WEB:DonnaSpencer} when conducting an exploratory analysis, is to standardise the labels for the groups. This is done by looking at the groups in \autoref{Tab:GroupOverview} and look for names that are similar or represent the same idea. The purpose of this is to make it easier to compare the groups a card belongs to. The standardised labels, which is found for this card sort, are listed in \autoref{StandardizedLabels}. Choosing the standardised labels is a balance, because it at one hand is a great tool for comparing the data and on the other hand, it may result in uniting groups where the subjects have had different thoughts. The \textit{Forum} group were found to draw many similarities to the other \textit{Community} labels, and it was therefore given the same label (see \autoref{StandardizedLabels}) \\ 

\begin{table}[H]
\begin{tabular}{|l|l|l|l|}
\hline
\rowcolor{xGray25} \textbf{Community} & \textbf{Search} & \textbf{TonePrints} & \textbf{Pedals} \\ \hline
Community & Finding TonePrints & Tone Prints & Pedals \\
Forum & TonePrint Search & TonePrints & Products \\
Community & & & \\
TonePrint Sharing/Community & & & \\ \hline
\rowcolor{xGray25} \textbf{Profile} & & & \\ \hline
Personal Profile & & & \\
User Profile & & & \\ \hline
\end{tabular}
\caption{The Standardized labels are seen in the top row, with the groups belonging to each seen below}
\label{StandardizedLabels}
\end{table}

\newcolumntype{R}{>{\raggedright\arraybackslash}X} %Left adjust and no stretch of text. Good for narrow cells
\begin{table}[h]
\small
\begin{tabularx}{\textwidth}{|R|R|R|R|R|R|}
\hline
\rowcolor{xGray25} \textbf{Subject 1} & \textbf{Subject 2} & \textbf{Subject 3} & \textbf{Subject 4} & \textbf{Subject 5} & \textbf{Subject 6} \\ \hline
Sharing (29) & Customer service (3) & Tone Prints (24) & Community (6) & User feedback (6) & Very Important (14) \\ \hline
Community (8) & Social (6) & Pedals (7) & Products (3) & User Profile (8) & Somewhat Important (9) \\ \hline
Support (3) & Toneprint software (9) & Subscribtion (3) & Artists (4) & Tecnical aspect (7) & Neutral (12) \\ \hline
Hardware/ Tools (1) & Finding TonePrints (16) & Forum (9) & Download \& Tools (5) & TonePrint search (11) & Somewhat Unimportant (5) \\ \hline
Creating/ Editing (2) & User Settings (6) && TonePrints (11) & TonePrnt sharing / community (11) & Very Unimportant (3) \\ \hline
& Ratings (3) && Personal Profile (7) && \\ \hline
&&& User Feed (7) && \\ \hline
\end{tabularx}
\caption{The column shows the groups a subjects has created and how many cards the group consist of}
\label{Tab:GroupOverview}
\end{table}

\section{Group Analysis}
\label{GroupAnalysis}
%
At the first phase of the analysis, it's desired to analyze the groups created by the subjects. The scope is to explore what kind of groups, the subjects think of when being presented the cards, and how similar and different the subjects think. One of the first things that becomes clear is that one subject has created groups very differently compared to the other five. Subject 6 has chosen to group the cards after how important he thinks they are. Even though there may be derived some interesting findings from subjects 6's groups they aren't found comparable with the groups created by the other subject. After looking for tendencies in the other subjects' groups, it might be possible to compare the trends to what Subject 6 finds interesting. When looking at the remaining subjects there seems to be several similarities. One theme that may be shared by the groups is "social interaction and social media". This is clear from most of the subject making a \textit{community} group while the \textit{Social} group draws similarities to different community groups. Having this alongside two \textit{Profile} groups and \textit{User settings} it seems like the idea of a sharing platform also includes having a part of it allocated to yourself. \\

\noindent
There also are different groups allocated to "Finding TonePrints" where the \textit{Search} and the \textit{Pedal} groups draw many similarities by mainly focusing on library and filter cards. Here, the \textit{Search} is also focusing on other features for finding a Toneprint e.g 'TonePrint Description', 'Tags describing a TonePrint' and 'Searching system'. Another theme that has different takes is "Feedback". The \textit{User Feedback} group focusses on different aspects of feedback options, whereas half of the group consists of the identical groups \textit{Support} and \textit{Customer Service} which refer to communication options with TC and artists. For the \textit{Technical Aspect} group there is generally a scope of "functionalities that the app should be able to do concerning individual TonePrints". Five of the cards in this group are the same, which \textit{Download \& Tools} is based on, and by adding more to it, the group \textit{TonePrint Software} is formed. Narrowing it down, the group \textit{Creating / Editing} may be located, which only focusses on the concepts of adjusting parameter setting by yourself or giving your friends the chance. Also, the \textit{Hardware / Tool} group is present and is only consisting of the 'Beaming tool' card. The \textit{Subscription} group and the \textit{Artist} group are both small and are concentrated mainly on the cards in which the name of the groups are present. However, in each group there is a card that seems misplaced, because it doesn't seem to have any relation to the others, which is more likely to happen for larger groups. The remaining groups are the two \textit{TonePrints} groups and the \textit{Sharing} group. The two large groups consists of more than half of the cards each and are therefore difficult to interpret by themselves and their content, because it varies a lot. This also seems to be the case with the smaller \textit{TonePrints} group, which might have been used as a group for cards, the subject felt difficult to group. \\

\noindent
As mentioned, the groups formed by Subject 6 are very different than those created by the other subjects, because he grouped the cards after how important he think they are. This actually gives the interesting opportunity to look for tendencies regarding what he think is important, and the groups which already have been explored. When looking at the \textit{Very Important} group it's clear that it's focused on functionalities that on it's own would be a TonePrint app that works fine. There are only two cards in this group that also appears in a community group, where one of them only is placed there by a single subject and the other by two. This indicates that Subject 6 doesn't seem to find the community aspect very important. The groups which appears most in the \textit{Very Important} groups is the \textit{Search} groups and the \textit{TonePrint Software} and \textit{Technical Aspects} groups, without counting the two large groups that often appears because of their wide description. For the \textit{Somewhat Important} there is a bit more mix of which groups that appears. Almost half of the group is related to the \textit{Search} groups while the rest is a bit mixed, however, several \textit{Community} and other "Social interaction" groups does appear. The most frequent groups found in \textit{Neutral} are the community related groups, and when looking at the cards this group is based on, it's functionalities and concepts that generally describes a profile based sharing platform. There is one card that seems odd compared to the others, which is the 'Beaming Tool'. This card have more in common with the cards in the \textit{Very Important} in terms of function. The \textit{Somewhat Unimportant} group consists of "Filter" cards and cards regarding "information about or from other users". Four of the five cards is directly concerned with other users. The things that Subject 6 finds \textit{Very Unimportant} is all related to \textit{Community} groups. The cards are \textit{Link to other social media profiles}, \textit{"Other TonePrint users near you" suggestions} and \textit{Comment section}.

\section{Card Analysis}
\label{CardAnalysis}
When analyzing the results of a card sort, it's not only interesting to look at the groups created and named by the individual users, but also to look at how frequent some cards are grouped together. For this purpose, the data from Subject 6 has been excluded, because he grouped them so fundamentally different. When looking at the frequency of cards grouped together, a distance matrix is created. Each row of the distance matrix represents a card and the same goes for each column, creating a half matrix where each cell compare two different cards, resulting in each card being compared to one another. For every time, two cards have been grouped together, 1 is added to the cell value, which in our case have the cells varying between the values 5 to 0. When looking at which cards that always or often are grouped together, it's possible to get a general idea of which natural groups that exist in the cards, without being interpreted on the basis of a label or other cards in the larger groups. When looking at which cards that always are grouped together, seven groups are identified. The seven groups of cards that always appears together are displayed in \autoref{FrequentGroups}. As illustrated by the labels does each of the groups describe different functionalities, whereas some still have similarities e.g. the to Search related groups. However, the two groups that are closets to appear in together is the \textbf{TonePrint Related search} and \textbf{Popularity of TonePrint}. These groups doesn't seams to have much in common, however when looking at the groups these cards appear in, it's only one subject that doesn't place them in the same group. Here it's seen that subject 5 have placed the the \textbf{Popularity of TonePrints} cards in his category \textit{Community} and the \textbf{TonePrint related search} cards in his \textit{Search} group. This is a good indication of how the small sample size affects the analysis of the the card sort, because there one subjects changes might have a huge impact on the end result. 

\newcolumntype{R}{>{\raggedright\arraybackslash}X} %Left adjust and no stretch of text. Good for narrow cells
\begin{table}[h]
\begin{tabularx}{\textwidth}{|R|R|R|R|}
\hline
\rowcolor{xGray25} \textbf{Personal input} & \textbf{Social Media} & \textbf{Feedback with Feedback} & \textbf{Pedal related search}\\ \hline
User biography & Link to other social media profiles & Feedback from users to TC Electronic & 'Effect type' Filter \\
Description of personal equipment & Public forum & Feedback from TC Electronic to users & 'Pedal type' Filter \\ 
&&& Pedal Library \\ \hline
\rowcolor{xGray25} \textbf{TonePrint related search} & \textbf{Popularity of TonePrint} & \textbf{Upload / Download} & \\ \hline
Tags describing a TonePrint & "Other users also liked this" recommendation & Automatic cloud upload & \\
Toneprint description & 'Rating' Filter & Toneprint upload system & \\ 
TonePrint categories & 'Popularity' filter & Toneprint download system & \\
TonePrint library & & Automatic cloud download & \\ \hline
\end{tabularx}
\caption{Cards always grouped together}
\label{FrequentGroups}
\end{table}


%1\textbf{'Rating' Filter}, \textbf{"Other users also like this" Recommendation} and \textbf{'Popularity' Filter}\\
%2\textbf{Tags Describing a TonePrint}, \textbf{TonePrint Describtion}, \textbf{TonePrint Categories}, \textbf{Searching sytem} and \textbf{TonePrint Library} \\
%3\textbf{'Artist' Filter}, \textbf{Artist Library} and \textbf{'Genre' filter} \\
%4\textbf{'Pedal type' filter}, \textbf{'Effect type' filter} and \textbf{Pedal library} \\
%5\textbf{User ranking overview} and \textbf{User library} \\
%6\textbf{User's favorite genres}, \textbf{Description of personal setup}, \textbf{Description of personal equipment} and \textbf{User biography} \\
%7\textbf{Automatic cloud download}, \textbf{Automatic cloud upload}, \textbf{TonePrint download system}, \textbf{TonePrint upload system} and \textbf{Beaming tool}\\
%8\textbf{Public forum}, \textbf{Link to other social media} and \textbf{Private forum} \\
%9\textbf{"Subscribe to user" system} and \textbf{News feed from "subscribed to" users} \\
%10\textbf{Feedback from users to TC Electronic}, \textbf{Feedback from TC Electronics to users} and \textbf{Profecional artist feedback} \\



%Look at individual Cards, Thous often grouped together where are they)\\
%(Interpret in relating to IA)




%For the remaining subjects they all have created  a group indicating something "Social Media" related. The \textit{Community} and \textit{Social} groups consist of 19 different cards. A overall theme for the cards in these groups is that thy relates to some kind of communication with other people e.g. "Public Forum" or "Link to other social media profile". \\
\section{Use of Card sort data}
\label{UseOfCardSortData}
%
As mentioned earlier is the results of the card sort heavily affected by a very small sample size. One of the reasons for conducting the card sort online was to make it easier to reach real users and many, while planing and conducting a workshop. This limits the possibilities of collecting qualitative data about the reasoning behind the groups created by the individual subjects. The limitations of gathering qualitative data makes the analysis more reliant on the qualitative aspects of the data, which is problematic to interpret while have a very small sample size. \\
What can be derived from the analysis \autoref{GroupAnalysis} is that the subjects mainly agree on dividing the cards in groups focused on Social interaction like that of social media, in which users have a profile. Groups focused on finding TonePrints which may be done by filtering a search in different libraries. There also appear groups focusing on communication in the form of feedback either to or from TC electronic and Professional artists. A group that also is seen is focus on having having somewhat basic functionalities for a TonePrint sharing platform that enables uploading, downloading and parameter adjusting. these tendencies is also to some extend apparent when Subject 4 and 5 put words to how they grouped the content.\\

\subsection{Lesson learned}
\label{LessonLearned}
Clearly hasn't this card sort been a success. The small sample size and the inclusion of very large groups have hindered interpretation of the groups created between subjects. The groups in which the cards are paired every time might give some indications, however when moving one layer out there is many groups that get entangled. Having not fulfilled the purpose of the card sort a new one might be conducted. This time it would it may be beneficial to conduct it in person, even though it might be difficult to find a large sample size of real users. On the other hand would the amount of qualitative data probably outweigh the sample size problem, or the "Realness" necessary for subjects. 







