\chapter{Card Sort Analysis}
\label{CardSortAnalysis}
%
There are two overall branches of analyzing the results from a card sort experiment, Exploratory or Statistical and in some cases both are used, to extract as much information as possible from experiment. The type of analysis is largely dependent on the goals of the experiment. The goal for the card sort conducted in this study have been to explore the general idea of how the users of the TonePrint system group concepts of the TonePrint sharing platform. This both includes which kind of groups they create and which concepts they find to belong together. The sample size of the data is also a factor when analyzing the data, because it may be difficult to derive meaningful statistics from a small sample size. \\
For the card sort conducted in this study 17 have opened the online test whereas few have started and left and One have completed half of it, but only Six have completed the card sort. Only the Six complete data sets are included in the analysis.

\section{Summary of Results}
\label{SummaryResults}
%
Six subjects have completed the card sort from which all were male aging from 26 to 44 years old with an average at 34.7 years, and they all come from different countries. Through the card sort the subject have constructed a combined total of 32 groups of varying sizes which may be seen in \autoref{Tab:GroupOverview}. \\ 
Before starting the analysis it's desired to create an overview of how the individual cards have been sorted. This is done in a spreadsheet (Lav billag og referere til det) containing all the cards in one column with the names of the groups it's placed in the adjacent columns. The next step as described by \textcite[184]{WEB:DonnaSpencer} when conducting an exploratory analysis is to standardize labels for the groups. This is done by looking at the groups seen in \autoref{Tab:GroupOverview} and look for names that are close to similar or represent the same idea. The purpose of this is to make it easier to compare the groups a card belongs. The standardized labels which is found for this card sort are listed in \autoref{StandardizedLabels}. Choosing the standardized labels is a balance because it at one hand is a great tool for comparing the data and on the other hand it may result in uniting groups where the subjects have had different thoughts. The \textit{Forum} group were found to draw many simmilarities to the other \textit{Community} labels and is was therefore given the same label \autoref{StandardizedLabels} \\

\begin{table}[H]
\begin{tabular}{|l|l|l|l|}
\hline
\textbf{Community} & \textbf{Search} & \textbf{TonePrints} & \textbf{Pedals} \\ \hline
Community & Finding TonePrints & Tone Prints & Pedals \\
Forum & TonePrint Search & TonePrints & Products \\
Community & & & \\
TonePrint Sharing/Community & & & \\ \hline
\textbf{Profile} & & & \\ \hline
Personal Profile & & & \\
User Profile & & & \\ \hline
\end{tabular}
\caption{The Standardized labels are seen in the top row, with the groups belonging to each seen below}
\label{StandardizedLabels}
\end{table}

\newcolumntype{R}{>{\raggedright\arraybackslash}X} %Left adjust and no stretch of text. Good for narrow cells
\begin{table}[h]
\small
\begin{tabularx}{\textwidth}{|R|R|R|R|R|R|}
\hline
\textbf{Subject 1} & \textbf{Subject 2} & \textbf{Subject 3} & \textbf{Subject 4} & \textbf{Subject 5} & \textbf{Subject 6} \\ \hline
Sharing (29) & Customer service (3) & Tone Prints (24) & Community (6) & User feedback (6) & Very Important (14) \\ \hline
Community (8) & Social (6) & Pedals (7) & Products (3) & User Profile (8) & Somewhat Important (9) \\ \hline
Support (3) & Toneprint software (9) & Subscribtion (3) & Artists (4) & Tecnical aspect (7) & Neutral (12) \\ \hline
Hardware/ Tools (1) & Finding TonePrints (16) & Forum (9) & Download \& Tools (5) & TonePrint search (11) & Somewhat Unimportant (5) \\ \hline
Creating/ Editing (2) & User Settings (6) && TonePrints (11) & TonePrnt sharing / community (11) & Very Unimportant (3) \\ \hline
& Ratings (3) && Personal Profile (7) && \\ \hline
&&& User Feed (7) && \\ \hline
\end{tabularx}
\caption{The column shows the groups a subjects has created and how many cards the group consist of}
\label{Tab:GroupOverview}
\end{table}

\section{Group Analysis}
\label{GroupAnalysis}
%
At the first phase of the analysis it's desired to analyze the groups created by the subjects. The scope is to explore what kind of groups the subjects think of, when presented for the cards provided for this card sort and how similar and different the subjects thinks.\\  
One of the first thing that becomes clear is that one subject is have created created groups very differently than the other five subject. Subject 6 has chosen to group the cards accordingly to how important he think it is. Even though there may be derived some interesting findings from subjects 6's groups they aren't found comparable with with the groups created by the other subject. After looking for tendencies in the other subjects groups, it might be possible to compare the trends to with what Subject 6 finds interesting. When looking at the remaining subjects there seams to be several similarities. One theme that may be reflected by the groups is "social interaction and social media". This is clear by most of the subject making a \textit{community} group while the \textit{Social} group draws similarities to different community groups. Having this alongside two \textit{Profile} groups and \textit{User settings} it seams like the idea of a sharing platform also includes having a part of it allocated to yourself. \\
There also is different aspects groups allocated to "Finding TonePrints" where the \textit{Search} and the \textit{Pedal} groups draws many similarities by mainly focusing on library and filter cards. Here the \textit{Search} is also focusing on other features for finding a Toneprint e.g 'TonePrint Description', 'Tags describing a TonePrint' and 'Searching system'.  \\
Another theme that has different takes is that of "Feedback". The \textit{User Feedback} group focus on different aspects of feedback options, whereas half of the group consists of the identical groups \textit{Support} and \textit{Customer Service} which refer to communication options with TC and artists. \\
For the \textit{Technical Aspect} group there is a generally scope of "functionalities that the app should be able to do concerning individual TonePrints". Five of the cards in this group is the same thous which \textit{Download \& Tools} is based on, and by adding to more to it there group \textit{TonePrint Software} is formed. Narrowing it down the group \textit{Creating / Editing} may be located, which only focus on the concepts of adjusting parameter setting by yourself or giving your friends the chance. Also the \textit{Hardware / Tool} group is present and is only consisting of the 'Beaming tool' card.\\
The \textit{Subscription} group and the \textit{Artist} group are both small and are concentrated mainly on the cards in which the name of the groups are present. However is there in each group a card that seam misplaced, because it doesn't seam to have any relation to the others, which is more likely to happen for larger groups.\\
The remaining groups are the two \textit{TonePrints} groups and the \textit{Sharing} group. The two large groups consists of more than half of the cards each and is therefore difficult to interpret by them self and their content, because it varies a lot. This also seams to be the case with the smaller \textit{TonePrints} group, which might have been used as a group for cards the subject felt difficult to group. \\
\\
As mentioned is the groups formed by Subject 6 very different than those created by the other subject because he has grouped the cards in order of how important he think they are. This actually gives the interesting opportunity to look for tendencies regarding what he think is important and the groups which already have been explored. When looking at the \textit{Very Important} group it's clear that it's focused on functionalities that by it's own would be an TonePrint app that works fine. There is only two cards in this group that also appears in a community group, whereas one of them only is placed there by a single subject and the other by two. This indicates that Subject 6 doesn't seams to find the community aspect Very important. The groups which appears most in the \textit{Very Important} groups is the \textit{Search} groups and the \textit{TonePrint Software} and \textit{Technical Aspects} groups, whiteout counting the two large groups that often appears because of their wide description.\\
For the \textit{Somewhat Important} there is a bit more mix of which groups that appears. Almost half of the group is related to the \textit{Search} groups while the rest is a bit mixed, however does several \textit{Community} and other "Social interaction" groups appear.\\
The most frequent groups found in \textit{Neutral} is the community related groups, and when looking at the cards this group is based on it's generally is functionalities and concepts that generally describes a profile based sharing platform. There is one card that seams odd compared to the others, which is the 'Beaming Tool'. This card have more in common with the cards in the \textit{Very Important} in terms of function.\\
The \textit{Somewhat Unimportant} group consists of "Filter" cards and cards that regarding "information about or from other users". Four of the five cards is directly concerned with other users.\\
The things that Subject 6 finds \textit{Very Unimportant} is all related to \textit{Community} groups. The cards are \textit{Link to other social media profiles}, \textit{"Other TonePrint users near you" suggestions} and \textit{Comment section}.\\

\section{Card Analysis}
\label{CardAnalysis}
When analyzing the results of a card sort it's not only interesting to look at the groups created and named by the individual users, but also to look at how frequent some cards are grouped together. For this purpose the data from Subject 6 has been excluded because he have grouped them so fundamentally different. When looking at the frequency of cards grouped together a distance matrix is created. Each row of the distance matrix represents a card and the same goes for each column, creating a half matrix where each cell compare two different cards, resulting in each card being compared to one another. For every time two cards have been grouped together the 1 is added to the cell value, which in our case have the cells varying between the values 5 to 0. When looking at which cards that always or often is grouped together it's possible to get an general idea of which natural groups that exist in the cards without being interpreted on a basis of label an other cards in a the larger groups.\\
When looking at which cards that always is grouped together, seven groups is identified. The seven groups that of cards that always appears are seen in \autoref{FrequentGroups}. As seen by the labels does each of the groups describe different functionalities, whereas some still have similarities e.g. the to Search related groups. However, the two groups that are closets to appear in together is the \textbf{TonePrint Related search} and \textbf{Popularity of TonePrint}. These groups doesn't seams to have much in common, however when looking at the groups these cards appear in, it's only one subject that doesn't place them in the same group. Here it's seen that subject 5 have placed the the \textbf{Popularity of TonePrints} cards in his category \textit{Community} and the \textbf{TonePrint related search} cards in his \textit{Search} group. This is a good indication of how the small sample size affects the analysis of the the card sort, because there one subjects changes might have a huge impact on the end result. 

\newcolumntype{R}{>{\raggedright\arraybackslash}X} %Left adjust and no stretch of text. Good for narrow cells
\begin{table}[h]
\begin{tabularx}{\textwidth}{|R|R|R|R|}
\hline
\textbf{Personal input} & \textbf{Social Media} & \textbf{Feedback with Feedback} & \textbf{Pedal related search}\\ \hline
User biography & Link to other social media profiles & Feedback from users to TC Electronic & 'Effect type' Filter \\
Description of personal equipment & Public forum & Feedback from TC Electronic to users & 'Pedal type' Filter \\ 
&&& Pedal Library \\ \hline
\textbf{TonePrint related search} & \textbf{Popularity of TonePrint} & \textbf{Upload / Download} & \\ \hline
Tags describing a TonePrint & "Other users also liked this" recommendation & Automatic cloud upload & \\
Toneprint description & 'Rating' Filter & Toneprint upload system & \\ 
TonePrint categories & 'Popularity' filter & Toneprint download system & \\
TonePrint library & & Automatic cloud download & \\ \hline
\end{tabularx}
\caption{Cards always grouped together}
\label{FrequentGroups}
\end{table}


%1\textbf{'Rating' Filter}, \textbf{"Other users also like this" Recommendation} and \textbf{'Popularity' Filter}\\
%2\textbf{Tags Describing a TonePrint}, \textbf{TonePrint Describtion}, \textbf{TonePrint Categories}, \textbf{Searching sytem} and \textbf{TonePrint Library} \\
%3\textbf{'Artist' Filter}, \textbf{Artist Library} and \textbf{'Genre' filter} \\
%4\textbf{'Pedal type' filter}, \textbf{'Effect type' filter} and \textbf{Pedal library} \\
%5\textbf{User ranking overview} and \textbf{User library} \\
%6\textbf{User's favorite genres}, \textbf{Description of personal setup}, \textbf{Description of personal equipment} and \textbf{User biography} \\
%7\textbf{Automatic cloud download}, \textbf{Automatic cloud upload}, \textbf{TonePrint download system}, \textbf{TonePrint upload system} and \textbf{Beaming tool}\\
%8\textbf{Public forum}, \textbf{Link to other social media} and \textbf{Private forum} \\
%9\textbf{"Subscribe to user" system} and \textbf{News feed from "subscribed to" users} \\
%10\textbf{Feedback from users to TC Electronic}, \textbf{Feedback from TC Electronics to users} and \textbf{Profecional artist feedback} \\



%Look at individual Cards, Thous often grouped together where are they)\\
%(Interpret in relating to IA)




%For the remaining subjects they all have created  a group indicating something "Social Media" related. The \textit{Community} and \textit{Social} groups consist of 19 different cards. A overall theme for the cards in these groups is that thy relates to some kind of communication with other people e.g. "Public Forum" or "Link to other social media profile". \\
\section{Use of Card sort data}
\label{UseOfCardSortData}
%
As mentioned earlier is the results of the card sort heavily affected by a very small sample size. One of the reasons for conducting the card sort online was to make it easier to reach real users and many, while planing and conducting a workshop. This limits the possibilities of collecting qualitative data about the reasoning behind the groups created by the individual subjects. The limitations of gathering qualitative data makes the analysis more reliant on the qualitative aspects of the data, which is problematic to interpret while have a very small sample size. \\
What can be derived from the analysis \autoref{GroupAnalysis} is that the subjects mainly agree on dividing the cards in groups focused on Social interaction like that of social media, in which users have a profile. Groups focused on finding TonePrints which may be done by filtering a search in different libraries. There also appear groups focusing on communication in the form of feedback either to or from TC electronic and Professional artists. A group that also is seen is focus on having having somewhat basic functionalities for a TonePrint sharing platform that enables uploading, downloading and parameter adjusting. these tendencies is also to some extend apparent when Subject 4 and 5 put words to how they grouped the content.\\

\subsection{Lesson learned}
\label{LessonLearned}
Clearly hasn't this card sort been a success. The small sample size and the inclusion of very large groups have hindered interpretation of the groups created between subjects. The groups in which the cards are paired every time might give some indications, however when moving one layer out there is many groups that get entangled. Having not fulfilled the purpose of the card sort a new one might be conducted. This time it would it may be beneficial to conduct it in person, even though it might be difficult to find a large sample size of real users. On the other hand would the amount of qualitative data probably outweigh the sample size problem, or the "Realness" necessary for subjects. 







