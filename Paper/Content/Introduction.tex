\chapter{Introduction}
\label{Introduction}
The following master's thesis takes its starting point in the TonePrint concept from TC Electronic (from here on referred to as \textit{TC}). Its reveal in 2011 opened a new playground for musicians and tone tweakers, making effect editing possible from the users smartphone. Until this point, TC was already a worldwide known manufacturer of effect pedals for guitarists, originally formed in the early 1970's by Kim and John Rishøj in Aarhus, Denmark. With this as the baseline for the thesis, the focus will be on the future for this concept, starting with a general description of effect units and the capabilities of TonePrints.

\section{The TonePrint Concept}
\label{TonePrints}
Effect pedals in general are well known units for guitarists and bassists alike, spanding multiple music genres. The pedal works by taking the input signal from the guitar and changing it as to the user's tweaking. Tweaking can be of different parameters depending on the effect type, and when playing, the user activates these changes by a push of a single button on the pedal. An example of a simple guitar effect pedal is displayed on \autoref{fig:EffectPedalExample}, where the adjustable parameters on it consists of \textit{Dwell, Mix,} and \textit{Tone}. Each of these are accessed and tweaked with individual knobs on the unit, which gives the user a limited range of ways to change the sound.

With this limitation as a motivation, TC created the TonePrint concept, enabling users to tweak the sound of effects beyond the parameters on the pedals. Using the TonePrint application, the users have a vast selection of custom presets with further parameters available for tweaking. These presets are what the term \textit{TonePrint} covers and they are either created in collaboration with professional musicians or by the common user. In order to distinguish these from each other, they are referred to as \textit{Artist TonePrints} and \textit{User TonePrints} respectively. After selecting one for the effect pedal in question, the user can make any desired tweaking or transfer it directly to the pedal with the option of altering it even more on the physical knobs \parencite{PDF:TonePrintAnalyse}. TC has collaborated with multiple guitarists and bassists, creating TonePrints for effect pedals used by the artists themselves. After the creators are satisfied with their TonePrints, they are uploaded to the TonePrint library in the application where any users of the same effect pedal can download the TonePrint and as such match the sound of their favourite artist. For User TonePrints the overall concept is the same. They differ in the fact that the creator isn't a famous guitarist, but the TonePrint is still made using the application and can be transferred directly to its effect pedal. However, when it comes to sharing these User TonePrint with friends and other aspiring guitarist, a platform for this purpose doesn't exist yet.
%
\begin{figure}[H]
	\centering
	\includegraphics[width=.20\textwidth]{Graphics/EffectPedalExample}
	 \caption{This figure shows a Drip spring reverb effect pedal by TC Electronics \url{https://www.tcelectronic.com/Categories/Tcelectronic/Guitar/Stompboxes/DRIP-SPRING-REVERB/p/P0CQ2\#googtrans(en|en)}.}
    \label{fig:EffectPedalExample}
\end{figure}
%
\subsection{The TonePrint Software}
\label{TonePrintSoftware}
As previously stated, the exploring of TonePrints start with the TonePrint application available for smartphones and tablets. However, the software is also available for PC and MAC, and the reason for this distinction lies in the difference of how a TonePrint is transferred to its respective pedal. For PC and MAC the user is required to use a cable from the computer to the pedal, but through the tablet and smartphone application, the user also have the option of simply beaming it directly to the pedal. whatever the platform, however, when opening the software the user is introduced to a list selection of different effect pedals, each holding a vast number of TonePrints created by famous guitarists. After selecting an effect pedal from this list, the user is then presented a new list selection of the many guitarist who have created TonePrints for this pedal. When selecting one of the guitarists, and depending on whether the guitarist have created more TonePrints for the same pedal, the user is then presented a bigger view of this specific TonePrint with a description of it and its creator. An example of this is displayed on \autoref{fig:TonePrintAppExample}. Depending on the users' motivation when opening the application first time, they can also choose to browse by artist instead of pedal, if their starting point is to find out what it takes to sound like their favourite artist.

\begin{figure}[H]
	\centering
	\includegraphics[width=\textwidth]{TonePrintAppExample}
	\caption{The view in the TonePrint application after selecting an effect pedal and a TonePrint. This example displays a TonePrint created by Johan Wohlert of the danish rock band \textit{Mew}.}
	\label{fig:TonePrintAppExample}
\end{figure}
\noindent

\begin{itemize}
	\item Man kan overføre med kabel eller beaming\fxnote{Der skal laves henvisning til appen i sig selv}
	\item If a user should attempt to transfer a TonePrint of an effect pedal to a completely different unit, it simply wouldn't work.
	\item Et billede af appen, men hvad skal det billede helt præcist vise? Hvor i appen skal man være?\fxnote{Indsæt billede af appen}
	\item Brugeren kan også starte med at browse ud fra guitarist i stedet for deal type => Det er ikke til at sige, hvad brugerens motivation er.
\end{itemize}

%\section{What should be in the introduction}
%\label{WhatShouldBeInTheIntroduction}
%
%\begin{itemize}
%	\item A short description of TC Electronic (Just the basics)
%	
%	\begin{itemize}
%		\item TC laver effektpedaler og er kendt på verdensplan
%		\item  A very short description of the principal of TonePrint (A later chapter/section will go more in depth \autoref{TonePrints}
%		\item Something about TC wanting to enable the users to share their UserTonePrints
%	\end{itemize}
%	
%	\item TC wish to include their users in the development of the new sharing platform (Community).
%	
%	\begin{itemize}
%		\item Hvorfor er det generelt en god ide at inddrage brugere til at løse designproblemer?
%		\item Hvad er typiske problemer?
%	\end{itemize}
%	
%	\item Something in general about how communities works and why it's important to include the users in the development of one.	
%	
%	\begin{itemize}
%		\item TC's Koncern community
%		\item Generelt ønske fra brugerne, at de gerne vil have et TonePrint community. - Det kræver kun et besøg på diverse fora for at finde dette ønske
%	\end{itemize}
%
%\end{itemize}


%Processen skal gå fra at TC har ønske om at lave et TonePrint Community. Det vi så skal arbejde hen imod er at finde ud af, hvordan en virksomhed som TC kan foretage brugerinddragelse, og hvordan de specifikt skal gøre det i forhold til dette.



%
%\section{The TonePrint concept}
%\label{TonePrints}
%In 2011 TC Electronic revealed their first TonePrint pedals, opening a new playground for musicians and tone tweakers. By applying the smartphone as tool for setting the desired parameters for an effect type, the user can beam a preset from his or her smartphone directly to the pedal holding the effect type in question. The number of parameters varies between pedal types, and each setting of these parameters is what is simply referred to as \textit{TonePrints} \parencite{WEB:AboutTonePrints}.
%
%TC Electronic's application holds multiple templates made in-house with different effect types and styles, but it also holds multiple TonePrints created in collaboration with famous guitarists. These presets are referred to as \textit{Artist TonePrints}, and the expectation is that the user will have a clear expectation of the sound of the TonePrint, when it comes from his or her favourite guitarist \parencite[][8]{PDF:TonePrintAnalyse}. As such, the user has the option of either using a specific effect type as a starting point or having the desire of matching the sound of a famous guitarist. In either case the TonePrint is beamed to the effect pedal in question, allowing for further tweaking of the sound before it is send through the rest of the setup and perceived from the speaker. The user may also want to explore TonePrints from guitarists unknown to them and maybe influence his or hers way of playing \parencite[][8]{PDF:TonePrintAnalyse}.
%
%Despite the various templates and artist TonePrints, the motivation of the users may also be to develop their own unique sound from scratch, these presets are referred to as \textit{User TonePrints}. The user also manages these in the TonePrint app, but at its current stage these TonePrints are still constrained to reside in the user's app without a straightforward way of sharing the settings with other guitarists. The next step for the TonePrint concept therefore seems to be a community in some form, allowing users to share their unique TonePrints with each other.

%Det ovenstående afsnit børe dummes ned så det kan forståes af folk som ikke er guitarister eller kender til TC's produkter.
%TonePrint Editoren skal også beskrives.
%Nogle billeder og eksempler ville ikke være dumt.
%Hvis det ikke skal stå i introduktionen kan vi også komme mere i dybten med det. TonePrint er en vigtig del i vores opgave da det er dem som brugerne skal dele.


%Hvad skal der med i afsnittet om TonePrints?
%- Et unikt preset af en effekttype
%	- Disse effekttyper er tilknyttet deres egne pedaler, hvor TonePrints er forskellige indstillinger af disse effekttyper
%	- (Indstillingerne af effekten omfatter 50 - 100 parametre, afhængigt af pedal model)
%	- Pedalerne kan have mange forskellige TonePrints tilhørende sig, disse er skabt og tilpasset i samarbejde med stjerneguitarister
%- Intentionen med at lave TonePrints i samarbejde med kendte guitarister er at give brugeren en klar ide om, hvad de kan forvente af disse presets.
%	- En brugers udgangspunkt vil måske også være at lyde som sit absolutte idol.
%	- Alternativt, kan man også opdage TonePrints fra guitarister, hvis stil man ikke kender, og på den måde få udvidet sin horisont.


%		- Forventningsbekræftende vs forventningsbrydende
%	- TC's række af samarbejdspartnere er stødt stigende
%- Artist TonePrints vs User TonePrints
%	- Efter, man har taget udgangspunkt i et TonePrint, kan man foretage yderligere ændringer af det og på den måde gøre det til sit eget (User TonePrints).
%	- Artist TonePrints eller TC-Templates
%- Computer app vs tablet app
%	- Beam direkte fra smartphone og ned i pedalen