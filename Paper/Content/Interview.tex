\chapter{Interview with the TonePrint app development team}
\label{Interview}
\begin{LARGE}
Interview with the development team
\end{LARGE}

%\begin{itemize}
%	\item Aim of questions
%		\begin{itemize}
%			\item Describe the focus points of the interview (I think we had four areas of interest)
%		\end{itemize}
%	\item Questions
%		\begin{itemize}
%			\item Questions as result of the evaluation
%			\item The rest of the questions and each questions purpose
%			\item Interview guide.
%		\end{itemize}
%	\item Setup
%		\begin{itemize}
%			\item The experimental description
%			\item Describe how it went
%			\item Short description of the results and how they are handled (Something leading into the thematic analysis)
%		\end{itemize}
%	\item Thematic analysis
%		\begin{itemize}
%			\item What is the purpose?
%			\item How is it done
%			\item How we have done it
%		\end{itemize}
%	\end{itemize}
%
Through \autoref{AnalyzingTonePrintDesign} it's decided to conduct a semi-structure interview focused on evaluating TC's design process of the TonePrint app. The information derived from this interview will than serve as a base when helping shape the design process of the TonePrint community towards a user centered approach. In the effort of trying to keep the purpose of the interview in mind four interview research questions is presented in \autoref{PreparingTheInterview}. When designing the interview each question should serve the purpose of help answering one of the four interview research questions, in order to derive the desired information. Furthermore does the knowledge obtained by exploring the TonePrint app through an heuristic evaluation help framing the predetermined questions towards being context specific. It also provides the knowledge necessary to have a conversation with the developers in terms of having them elaborating their answers and digging into more specific areas of the app during the conversation. 
%
\section{Designing the interview guide}
\label{InterviewGuide}
When planing the semi-structured interview a list of questions have to be prepared which purpose is to ensure that the interview research questions is answered. The interview is conducted in Danish because the team members of the development team are danish and even though conducting it in English might make the results easier to include in this project, might a potential languages barrier could make it more difficult for the interviewees to elaborate their answers. \\
\\
The first of the interview research questions is "\textbf{What is the TonPrint apps developers understanding of their target users and how was it applied in the design of the App?}. Two examples of questions that might help answer this question might be: '\textit{How have your knowledge of your users affected the development of the TonePrint app, and where do you have this knowledge from}' and '\textit{What type of feedback do you receive regarding the TonePrint editor and how do you use this feedback?}'. Even though the two questions differ quit a bit, they both provides information that is relevant for the interview research question. The first questions is directly asking if their user knowledge (understanding of the user) have affected the development, which is almost similar to the Interivew research question. Depending on the answer may the interviewer ask follow up question that could dig deeper into how the knowledge have affected a specific part of the app. When the second question focus on the feedback received it might enlighten how the developers create their under standing of the users.\\
\\
The second interview research question states "\textbf{How was the decision process when deciding on, and designing new implementations for the TonePrint App?}". Examples of questions answering this is: '\textit{How did you decide the information structure of the TonePrint app, regarding both the menu structure and the different ways they may be cattegorize}' and '\textit{What is the reasoning for the differences of the app given the platform used? e.g information of not having a pedal connected, the search function and how the video is played}'. Both of these examples are searching for information about how decisions were made though the design process. They are focusing on quite different ares, where the first focus on a general decision regarding the app, and the second focus on specific differences between platforms which were spotted though the exploration of the app. \\
\\
The third interview research question is "\textbf{How did the SCRUM framework affect the design process of the TonePrint app.}", and an example of a question highlighting this is '\textit{What positive and negative effects has the use of the SCRUM framework had on the development of the TonePrint app?}'. This question is very direct in terms of answering the interview research question, and in itself isn't it very context specific. However when conducting the interview as semi-structured the interviewer might direct the question towards some more specific examples which have been highlighted earlier in the interview.\\
\\
The fourth and final interview research question states "\textbf{Which aspects does the development team find important for the design process of the TonePrint community?}", which the following example tries to answer 'If you should mention five important aspects that we should focus on when developing the TonePrint community, what should it be?'. This question enables allows a conversation between the interviewer and the interviewee to begin, because it's framed as a the interviewer searching for help.\\
\\
It's important to remember that the interview is a semi-structured interview which means that the list of questions prepared for the interview serve more as a check list than a chronological step by step guide. It's not necessarily important that the questions is asked in a certain order, and the interviewer should be ready to ask follow up questions based on the answers to further elaborate. If the interviewer finds that one of the prepared questions already have been addressed through earlier answers and conversations it's not necessary to make the interviewee repeat themselves. \\
The interview guide is located in \autoref{AppendixInterviewGuide} and consists of 17 questions in danish and a short introduction that the read for the interviewees. The interview is planed to take approximately one hour, it may vary from time to time depending on the amount of answers each interviewee will provide.

\subsection{Conducting the interview}
\label{ConductingInterview}
%
The interview with TC was planed through the company supervisor of this project, who also help find relevant people to interview. The interview were conducted in a meeting room at TC's headquarter in Århus, doe to practicality.\\
Four interviewees with different roles participated in the interview, whereas the company supervisor were one of them. The other three was respectively a graphical designer, software programmer and product owner. The length of the interviews fluctuated between 23 minutes and on hour, and both authors of this project were participating in the interview. Each interview were audio recorded and transcribed for the purpose of analyzing the data. The transcribed interviews is located in the digital appendix \autoref{DigitalAppendix}.\\


\section{Method for analyzing the interview}
\label{InterviewAnalysis}
%
As a result of choosing to do a semi-structured interview, the interviews have taken different shapes according to the field of knowledge and interest of the individual interviewees. In order to meet these differences it's decided to use the qualitative analysis method 'Thematic analysis'. The scope of a thematic analysis is to organize the data gathered through e.g an interview in themes, explaining patterns or other interesting aspects of the data\parencite{PDF:Braun2006}. The themes created through thematic analysis are intended to make the interpretation of data scattered across ones data set easier. 

\subsection{Thematic analysis approach}
\label{ThematicAnalysisApproach}
When describing the phases of a thematic analysis in \parencite[86]{PDF:Braun2006} as a step-by-step guide it's emphasized that it is just a guide, and not a set of rules because it would counter the flexibility that this method provides. It's described that there is six phases for the thematic analysis, which is listed in \autoref{ThematicPhases}. These phases embodies all of the process from going through collecting the data, to describing the findings of the analysis.

\begin{table}[H]
\centering
\begin{tabular}[width=\textwidth]{cl}
\hline
& Phases of Thematic analysis \\ \hline
1 & Familiarizing yourself with your data \\
2 & Generating initial codes \\
3 & Searching for themes \\
4 & Reviewing themes \\
5 & Defining and naming themes\\
6 & Producing the report \\ \hline
\end{tabular}
\caption{The six phases of a Thematic analysis as depicted in \parencite[87]{PDF:Braun2006}}
\label{ThematicPhases}
\end{table}

The first phase which focus on getting familiar with the data of the data starts when the interviews are being transcribed. When transcribing the interview some interesting things might be found or patterns might start taking form. Even though transcription just is a way of transforming one's data it will also help providing a better insight and understanding of the data.\\
The second phase of the analysis is to code the data transcribed through the first phase. The process of coding is that extracts of the data that are found interesting is marked and given a code that in few words indicate the content of the data extract. This results in a long list of highlighted parts extracts of the transcribed interview with associated codes. \\
In the third phase the codes derived from the second phase is used to form themes. In this process the different codes are combined across the data set forming themes that highlights interesting aspects or patterns in the data. This results in different lists of data extracts from different parts of the interview or different subjects, which combined constitute a theme of the answers from the interview.\\
The fourth phase of the analysis is a reviewing phase, where all the themes are revisited. In the process of shaping the themes in the third phase it might be difficult to retain an overview of the content of the different codes and in different themes. This may result in themes containing codes describing to different content to keep them in same theme, or themes may be so similar that they may be combined. The purpose of the fourth phase is to revisit the themes in order to ensure that they make sense. \\
The scope of the fifth phase is focused on naming and describing the themes which have been identified through the earlier phases. Here the content of the individual themes is described in order to be used in the overall analysis of the interview.\\
The sixth and final phase is the writing part of the analysis where the obtained knowledge in the shape of themes is used to accommodate the purpose of the interview.\\



%Som beskrevet i \autoref{HeursiticEvaluation} har vi valgt at lave et interview for at undersøge udviklings processen ved TC. Ud fra den heuristiske evalueringer har vi fået en forståelse for nogle problemer ved TonePrint appe, hvilket giver grundlag for nogle af spørgsmålende i det følgende interview.
%As it is presented in the project proposals for this master's thesis, the focus is on investigating how a company such as TC Electronic can apply user inputs for the design of future products, before applying this knowledge on the specific case of the TonePrint community. Such an investigation firstly requires a look into the design process of the development teams at TC Electronic, in order to assess how and to what extend user involvement can be applied.

%Acquiring inputs from the targeted users of a product is in general considered beneficial for its development, as these inputs can help, the developers better understand the users needs and requirements. This enables them to counter otherwise problematic design choices. When users interact with new products they form a mental model of how the interaction should be done through their existing knowledge, general practice, and the provided instructions. The most ideal thing though would be to talk directly with the designer of the product, but since this isn't an option, they must rely on the available information from the product itself \parencite[][31]{PDF:DonNorman}. If the mental model of the users and the conceptual model of the developer don't match, it isn't guaranteed that the users will interact easily with the product.

%Involving the users in the design process may as such solve this issue, and depending on what the focus of such user involvement is, there are numerous ways of doing it. For the planning process, it is desired to know the possibilities and limitation for the various factors such as the overall focus for the involvement, possible time limitations, the number of available users, etc. and these may be influenced by the general design process at TC Electronic. In order to map this design process, interviews with different members of the TC staff will be conducted, and as such get a better understanding of any possibilities and limitations of user involvement at TC.


%Firstly it seems relevant to ask TC what they're hoping to acquire from user involvement in general, more specifically, what kind of information are they hoping to get from the users? Questions concerning this could be:
%
%\begin{itemize}
%	\item Når i udvikler, hvordan inkorporerer i så jeres viden om brugerne?
%	\item Er der nogle informationer om jeres "target users" som i føler i mangler, når i udvikler produkter?
%	\item Er det forskelligt fra produkt til produkt, hvilken information om brugerne i har brug for, og i så fald hvad er forskellen?
%\end{itemize}
%
%\noindent
%In correlation with this, it also seems relevant to focus on TC's own understanding of their target users. How do they apply their knowledge of them in the design process, and what factors play a role in this, such as genre, geography, economy, gender, etc. Questions concerning this could be:
%
%\begin{itemize}
%	\item Den nuværende viden i har om jeres brugere, hvordan har i tilegnet jer den?
%	\item Hvordan spiller forskellen på brugergrupper ind, når i udvikler produkter?
%		\begin{itemize}
%		\item Tilpasser i produkterne til forskellige brugergrupper? (geografi, køn, osv.)
%		\item Hvordan målretter i jeres produkter mod den ønskede slutbruger
%		\end{itemize}
%	\item Har i tidligere gjort brug af brugerinddragelse som interviews, spørgeskemaer, workshops eller lignende, og hvad er jeres erfaring med det?
%\end{itemize}
%
%\noindent
%Next the focus should be on the organisation of TC, in order to find out when users can be involved. By asking about what tools they use in their developing process (both their planning and execution of these) it will be easier to plan when eventual user studies can be conducted and how cumbersome and time consuming they can be. Questions concerning this could be:
%
%\begin{itemize}
%	\item Hvor fast er fremgangsmåden for jeres designprocesser?
%	\item Hvad er den typiske procedure for jeres udvikling af produkter?
%	\item Er der forskel på, hvordan i udvikler produkter alt efter hvilken type det er?
%	\item Hvordan har kendskabet til domænet indflydelse på jeres udvikling/design af produkter?
%	\item anvender i nogle bestemte metoder/værktøjer i jeres arbejdsproces?
%	\item Hvordan planlægger i udviklingsprocessen for et produkt?
%\end{itemize}
%
%\noindent
%Finally, the goals and demands that TC set for their products should be explored. How are these applied in the product development, and how do they measure whether they have been reached? This probably is close to the questions regarding who the users are, as a goal just as well may be that the product is user friendly. Nevertheless it is still considered of value to include. Questions concerning this could be:
%
%\begin{itemize}
%	\item Hvordan opstiller i krav og succeskriterier for jeres produkter?
%	\item Medregner i brugen i jeres krav? eller brugerne i jeres succeskriterier? hvis ikke, hvorfor s ikke?
%	\item Hvordan indsamler i feedback om jeres produkter?
%	\item Hvilke succeskriterier og krav ser i som de vigtigste for brugerne, og hvordan ville i kunne bruge feedback herom?
%\end{itemize}
%




%- "Der står sådan her i projektforslaget, derfor er vores intention at starte med det her" - Det skal nok i virkeligheden være ud fra et research question, men oh well..
%
%- For at finde ud af, hvordan brugerinddragelse kan gøres hos TC, er det nødvendigt først at forstå, hvordan arbejds/designprocessen fungerer hos TC.
%
%- Vores forslag er at tilegne os denne viden ved simpelthen blot at spørge dem i et interview-format. (Begrundelse i form af forventninger til det. Måske lidt kilder om interviews?)
%
%- Hvad er vores mål at få ud af interviews med TC, og hvem og hvor mange af dem, vil vi interviewe?
%
%- Hvad vil vi spørge ind til => Hvad er de konkrete spørgsmål?
%
%
%
%
%- It's important to gather information of the design and development process at TC electronics.
%	- This knowledge is important to be able to map which UX methods that are applicable.
%
%- How can we get the necessary information?
%	- By observing the process and analyze the different stages. (Tidskrævene og upræsis)
%	- By getting the information directly from the people that are involved.
%
%- With interviews we have the option to ask follow up questions, whenever there seams to be more information that could be relevant.
%
%- It may be difficult to verbalize how the all the processes are carried out, due to tacit knowledge and Transactive memory.
%
%- Which information is necessary to describe the design and development process.
%	- Does there exist a general model for analyzing a development process?
%	- What do we categorize as necessary knowledge and how do we decide?
