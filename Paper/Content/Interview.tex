\chapter{The Design process of TC Electronic}
\label{Interview}
%As it is presented in the project proposals for this master's thesis, the focus is on investigating how a company such as TC Electronic can apply user inputs for the design of future products, before applying this knowledge on the specific case of the TonePrint community. Such an investigation firstly requires a look into the design process of the development teams at TC Electronic, in order to assess how and to what extend user involvement can be applied.

%Acquiring inputs from the targeted users of a product is in general considered beneficial for its development, as these inputs can help, the developers better understand the users needs and requirements. This enables them to counter otherwise problematic design choices. When users interact with new products they form a mental model of how the interaction should be done through their existing knowledge, general practice, and the provided instructions. The most ideal thing though would be to talk directly with the designer of the product, but since this isn't an option, they must rely on the available information from the product itself \parencite[][31]{PDF:DonNorman}. If the mental model of the users and the conceptual model of the developer don't match, it isn't guaranteed that the users will interact easily with the product.

%Involving the users in the design process may as such solve this issue, and depending on what the focus of such user involvement is, there are numerous ways of doing it. For the planning process, it is desired to know the possibilities and limitation for the various factors such as the overall focus for the involvement, possible time limitations, the number of available users, etc. and these may be influenced by the general design process at TC Electronic. In order to map this design process, interviews with different members of the TC staff will be conducted, and as such get a better understanding of any possibilities and limitations of user involvement at TC.

\section{Interview with TC}
\label{InterviewInputs}

\begin{itemize}
  \item Da i udviklede konceptet for TonePrint editor appen, hvordan besluttede i hvilke funktioner der skulle være med og hvordan de skulle designes? \fxnote{Skal muligvis gøres mere konkret, så der bliver spurgt ind til konkrete eksempler inden for TonePrint.}
  \item Hvordan har jeres viden angående jeres brugere påvirket/formet udviklingen af TonePrint editoren? Og hvor har i den viden fra?
  \item Gjorde i noget for at målrette TonePrint editoren mod nogle bestemte brugere, og hvordan gjorde i det i såfald? \fxnote{Det kan godt være, man skal formulere det som "brugergrupper" i stedet.}
  \item Selvom TonePrint editoren er et ret unikt produkt har i så draget inspiration fra andre interne og eksterne produkter/systemer og i så fald hvordan?
\end{itemize}

\begin{itemize}
  \item Hvordan besluttede i jer for informationsstrukturen i TonePrint editoren både set i forhold til menustrukturen og de forskellige måder de kan kategoriseres? \fxnote{Hvad mener vi med "informationsstruktur"? Vi skal specificere det lidt mere. kom med eksempler så som browse by artist/product.}
  \item Hvad ligger til grunde for jeres valg om at have små differenceringer mellem Editor på computer, Iphone og android? Eksempelvis informationen om ikke tilsluttet pedal, søge funktionen, video visning og TonePrint information samt beaming?
\end{itemize}

\begin{itemize}
  \item Til hvilken grad bruger i informationer i får gennem TonePrint junkies facebook siden eller music tribe community? \fxnote{Nævn også youtube. TC er generelt ret gode til at svare kunder/brugere.}
  \item Meget har ændret sig fra de gamle editor og apps, til den nuværende editor app. Hvorfor ændrede i både den grafiske identitet og flere funktionaliteter?
  \item Hvad er den typiske feedback i får vedrørende TonePrint editoren og hvordan bruger i denne feedback?
  \item Hvilke positive og negative effekter har jeres SCRUM arbejdsmetode haft på udviklingen af TonePrint editoren?
  \item Hvilke teknologiske begrænsninger har i haft under udviklingen af TonePrint editoren og hvordan har i kompenseret for disse?
  \item Hvordan opstillede i kravene for TonePrint editor appen både konceptuelt, design, succes mæssigt? Og hvordan måler i om kravene er imødekommet? \fxnote{Lille martin vil gerne have den her delt op <3}
\end{itemize}
%Firstly it seems relevant to ask TC what they're hoping to acquire from user involvement in general, more specifically, what kind of information are they hoping to get from the users? Questions concerning this could be:
%
%\begin{itemize}
%	\item Når i udvikler, hvordan inkorporerer i så jeres viden om brugerne?
%	\item Er der nogle informationer om jeres "target users" som i føler i mangler, når i udvikler produkter?
%	\item Er det forskelligt fra produkt til produkt, hvilken information om brugerne i har brug for, og i så fald hvad er forskellen?
%\end{itemize}
%
%\noindent
%In correlation with this, it also seems relevant to focus on TC's own understanding of their target users. How do they apply their knowledge of them in the design process, and what factors play a role in this, such as genre, geography, economy, gender, etc. Questions concerning this could be:
%
%\begin{itemize}
%	\item Den nuværende viden i har om jeres brugere, hvordan har i tilegnet jer den?
%	\item Hvordan spiller forskellen på brugergrupper ind, når i udvikler produkter?
%		\begin{itemize}
%		\item Tilpasser i produkterne til forskellige brugergrupper? (geografi, køn, osv.)
%		\item Hvordan målretter i jeres produkter mod den ønskede slutbruger
%		\end{itemize}
%	\item Har i tidligere gjort brug af brugerinddragelse som interviews, spørgeskemaer, workshops eller lignende, og hvad er jeres erfaring med det?
%\end{itemize}
%
%\noindent
%Next the focus should be on the organisation of TC, in order to find out when users can be involved. By asking about what tools they use in their developing process (both their planning and execution of these) it will be easier to plan when eventual user studies can be conducted and how cumbersome and time consuming they can be. Questions concerning this could be:
%
%\begin{itemize}
%	\item Hvor fast er fremgangsmåden for jeres designprocesser?
%	\item Hvad er den typiske procedure for jeres udvikling af produkter?
%	\item Er der forskel på, hvordan i udvikler produkter alt efter hvilken type det er?
%	\item Hvordan har kendskabet til domænet indflydelse på jeres udvikling/design af produkter?
%	\item anvender i nogle bestemte metoder/værktøjer i jeres arbejdsproces?
%	\item Hvordan planlægger i udviklingsprocessen for et produkt?
%\end{itemize}
%
%\noindent
%Finally, the goals and demands that TC set for their products should be explored. How are these applied in the product development, and how do they measure whether they have been reached? This probably is close to the questions regarding who the users are, as a goal just as well may be that the product is user friendly. Nevertheless it is still considered of value to include. Questions concerning this could be:
%
%\begin{itemize}
%	\item Hvordan opstiller i krav og succeskriterier for jeres produkter?
%	\item Medregner i brugen i jeres krav? eller brugerne i jeres succeskriterier? hvis ikke, hvorfor s ikke?
%	\item Hvordan indsamler i feedback om jeres produkter?
%	\item Hvilke succeskriterier og krav ser i som de vigtigste for brugerne, og hvordan ville i kunne bruge feedback herom?
%\end{itemize}
%




%- "Der står sådan her i projektforslaget, derfor er vores intention at starte med det her" - Det skal nok i virkeligheden være ud fra et research question, men oh well..
%
%- For at finde ud af, hvordan brugerinddragelse kan gøres hos TC, er det nødvendigt først at forstå, hvordan arbejds/designprocessen fungerer hos TC.
%
%- Vores forslag er at tilegne os denne viden ved simpelthen blot at spørge dem i et interview-format. (Begrundelse i form af forventninger til det. Måske lidt kilder om interviews?)
%
%- Hvad er vores mål at få ud af interviews med TC, og hvem og hvor mange af dem, vil vi interviewe?
%
%- Hvad vil vi spørge ind til => Hvad er de konkrete spørgsmål?
%
%
%
%
%- It's important to gather information of the design and development process at TC electronics.
%	- This knowledge is important to be able to map which UX methods that are applicable.
%
%- How can we get the necessary information?
%	- By observing the process and analyze the different stages. (Tidskrævene og upræsis)
%	- By getting the information directly from the people that are involved.
%
%- With interviews we have the option to ask follow up questions, whenever there seams to be more information that could be relevant.
%
%- It may be difficult to verbalize how the all the processes are carried out, due to tacit knowledge and Transactive memory.
%
%- Which information is necessary to describe the design and development process.
%	- Does there exist a general model for analyzing a development process?
%	- What do we categorize as necessary knowledge and how do we decide?
