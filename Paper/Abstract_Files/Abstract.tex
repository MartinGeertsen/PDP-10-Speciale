\noindent
\textbf{Introduction}
A key quality parameter of 3D prints is the consistency of the visible layers \parencite{PDF:TextureBasedQualityAssessment}. While recent advances are made to improve the printing process itself \parencite{PDF:A3DPrintingPathOptimizerBasedOnChristofidesAlgorithm}, this project explores the possibility of a higher level of perceived quality through the use of three post-processing techniques: sanding, painting, and acetone vapouring. Each process attempts minimise the visibility of layers. Three objects, representing a wide range of physical characteristics, are selected for post-processing, on the basis of a picture sort experiment.
\blankline
\textbf{Objective}
Understanding how post-processing affects the quality assessment of 3D-printed objects. If either of the tested techniques prove successful it may unite fast printing with beautiful surface finish, at a reasonable price while the printing technology itself matures. 
\blankline
\textbf{Method}
Descriptive analysis was applied to obtain quality assessments of the objects. Unbiased individuals rated the objects on multiple attribute VAS one object at a time before rating the objects relative to their quality assessment. Attributes were obtained from a word elicitation, where a panel consisting of the group members came up with as many word as possible to describe the stimuli \parencite{PDF:DescriptiveAnalysis}. The resulting words were then sorted into groups, given descriptive headlines, and endpoints for their VAS. 36 subjects participated in the final test who were only presented one of the three objects effectively making it a between subject experiment. multiple presentations were implemented to counter carry-over effects, and the subjects were furthermore presented with different sequences \parencite{PDF:DescriptiveAnalysis}.. 
\blankline
\textbf{Results}
The correlation of quality and uniformity indicates uniformity as the most important attribute for quality assessment. A two way ANOVA further shows that they aren't significantly different from smoothness either. "Softness" does not affect the ratings which means it can be discarded. The twisted reference object is considered of the highest quality overall with significant higher ratings than the painted objects, the acetone vapoured sphere, and the the reference sphere
\blankline
\textbf{Conclusion}
This study demonstrates that the specific post-processing performed for the experiment does not improve perceived surface quality but qualitative data indicate that sanded objects have a comfortable feel to some subjects.